\documentclass[a4paper,twoside,DIV15,BCOR12mm,chapterprefix=true,headings=onelinechapter]{scrbook}
\usepackage{ana}

\usepackage{newunicodechar}
\usepackage{array}
\usepackage{longtable}

\newunicodechar{×}{\times}
\newunicodechar{∀}{\forall}
\newunicodechar{∃}{\exists}
\newunicodechar{∅}{\varnothing}
\newunicodechar{∩}{\cap}
\newunicodechar{α}{\alpha}
\newunicodechar{σ}{\sigma}
\newunicodechar{⇒}{\Rightarrow}
\newunicodechar{→}{\rightarrow}
\newunicodechar{↦}{\mapsto}
\newunicodechar{…}{\ldots}


\lecturer{Dr. C. Schmoeger}
\semester{Wintersemeseter 10/11}
\scriptstate{complete}

\author{Die Mitarbeiter von \url{http://mitschriebwiki.nomeata.de/}}
\title{Analysis III - Bachelorversion}
\makeindex

\begin{document}
\maketitle

\renewcommand{\thechapter}{\Roman{chapter}}
%\chapter{Inhaltsverzeichnis}
\addcontentsline{toc}{chapter}{Inhaltsverzeichnis}
\tableofcontents

\chapter{Vorwort}

\section{Über dieses Skriptum}

Dieses Dokument basiert auf dem Mitschrieb der Vorlesung \glqq Analysis III\grqq\ von Herrn Schmoeger im
Wintersemester 2010 an der Universität Karlsruhe (KIT). \\

Herr Schmoeger ist für den Inhalt nicht verantwortlich.

\section{Warum}

Diesen Dokument enthält alle relevanten Qefinitionen der Vorlesung. Dieses Dokument ist sozusagen die Header-Datei und das Skript  ist die Implementierung. Um die Vorlesung zu verstehen muss man alle Definitionen verinnerlichen. Es wird empfohle diese Dokument als Vorlesung auf die Vorlesung durchzuarbeiten.  

\section{Wer}
Gestartet wurde das Projekt von Joachim Breitner. Beteiligt an dem ursprünglichem  Mitschrieb sind Rebecca Schwerdt, Philipp Ost, Jan Ihrens, Peter Pan und Benjamin Unger.

Diese Kürzung wurde von Björn Jürgens erstellt.

\section{Wo}
Alle Kapitel inklusive \LaTeX-Quellen können unter \url{http://mitschriebwiki.nomeata.de} abgerufen werden.
Dort ist ein \emph{Wiki} eingerichtet und von Joachim Breitner um die \LaTeX-Funktionen erweitert.
Das heißt, jeder kann Fehler nachbessern und sich an der Entwicklung
beteiligen. Auf Wunsch ist auch ein Zugang über \emph{Subversion} möglich.


\renewcommand{\thechapter}{\arabic{chapter}}
\renewcommand{\chaptername}{§}
\renewcommand*{\chapterformat}{§\,\thechapter \enskip}
\setcounter{chapter}{-1}

\chapter{Vorbereitungen}
\label{Kapitel 0}
\textbf{Symbole in diesem Dokument:}

(nach Reihenfolge der ersten Verwendung)

Kapitel 0: 

$X,Y,Z$ Mengen ($\ne\varnothing$) \\
$f:X\to Y, g:Y\to Z$ Abbildungen. \\
$ x \in X, y \in Y, z \in Z$ \\
$A \subseteq X, B \subseteq Y, C \subseteq Z $

Kapitel 1:

Sei \(d\in\MdN\). (Dimension der betrachteten Intervalle)\\
$I_1,\ldots,I_d$ Intervalle in $\mdr$\\
$a=(a_1,\ldots,a_d), b=(b_1,\ldots,b_d)\in\mdr^d$\\

$\fa:$ $\sigma$-Algebra auf $X$\\
$\mu:\fa\to[0,+\infty]$ eine Abbildung. (i.d.R. ein Maß)\\

Kapitel 2:

\(\ci_{d}:=\{(a,b]\mid a,b\in\MdR^{d},\,a\leq b\}\) Menge der Halboffenen Intervalle


Kapitel 3:

Ab jetzt sei stets \(X\in\fb_{d}\). (Erinnerung: \(\fb(X)=\{A\in\fb_{d}\mid A\subseteq X\}\)) \\
Sei $M\subseteq\imdr$. \\
Sei $(f_n)$ eine Folge von Funktionen $f_n:X\to\imdr$.\\


Kapitel 4:

In diesem Paragraphen sei $\varnothing\ne X\in\fb_d$. Wir schreiben außerdem $\lambda$ statt $\lambda_d$.


Kapitel 6 

\(\varnothing\neq X\in\fb_{d}\)


Kapitel 7 

\(\varnothing\neq X\in \fb_d\).


Kabitel 8


In diesem Paragraphen seien \(k,l,d\in\mdn\) und \(k+l=d\). \(\mdr^d\cong\mdr^k\times\mdr^l\). Für Punkte \(z\in\mdr^d\) schreiben wir \(z=(x,y)\), wobei \(x\in\mdr^k\) und \(y\in\mdr^l\). \\
Sei \(C\subseteq\mdr^d\).


\newpage
\textbf{Schreibweisen / Bekanntes:}

\begin{tabular}{ p{3cm} p{2cm} p{4cm} p{6cm} }
  \textbf{Name} & \textbf{Symbol} & \textbf{$:=$ Definition} & \textbf{in Worten/Kommentar} \\ 
  \hline
  Folge von Mengen & $ (A_i) $ & $ $ & $(A_i)$ ist eine Folge in $\mathcal{P}(X)$ \\
  \hline
  disjunkte Vereinigung&  $\dot{\bigcup}_{j=1}^\infty A_j $ & $\bigcup_{j=1}^\infty A_j $ & Wie normale Veneinigung, nur sind die Ausgangsmengen disjunkt \\
  \hline
  Komplement  & $ A^c $ & $X\setminus A$ & Bezugsmenge X ist implizit \\ 
  \hline 
  Urbild & $f^{-1}(B) $ & $\{x\in X: f(x)\in B\} $ & falls $f$ bijektiv, dann ist $f^{-1}$ auch die Umkehrfunktion \\
  \hline 
  & $ \sum a_j$ & $\sum_{j=1}^\infty a_j $ & \\
  \hline
  offene Menge & $ $ & $ $ & todo… \\
  \hline
  offenes Intervall & $ $ & $ $ & todo… \\
  \hline
  geschlossen & $ $ & $ $ & todo…\\
  \hline
  äbgeschlossen & $ $ & $ $ & todo…\\
  \hline
  & $I_1 × I_2$ & $ $ & todo… \\

\end{tabular}

\textbf{Definitionen:}

\begin{tabular}{ p{3cm} p{2cm} p{4cm} p{6cm} }
  \textbf{Name} & \textbf{Symbol} & \textbf{$:=$ Definition} & \textbf{in Worten/Kommentar} \\ 
  \hline
  Potenzmenge & $\mathcal{P}(X)$ & $\{A:A\subseteq X\}$ & Menge aller Kombinationen \\
  \hline
  disjunkes \newline Mengensystem & $ $ & $∀ A,B \in P(\fm), A \ne B :A ∩ B = ∅$ & Die Menge von Mengen $\fm$ ist disjunkt, wenn alle ihre Untermengen disjunkt sind \\
  \hline
  disjunkte Folge & $ $ &  & analog zu Mengensystem \\
  \hline
  & $\mathds{1}_A(x)$ & $ \begin{cases}1, x\in A\\ 0, x\in A^c\end{cases} $ & \\ 

\end{tabular}


\chapter{$\sigma$-Algebren und Maße}
\label{Kapitel 1}


\textbf{Definitionen}:

\begin{longtable}{ m{3cm} m{1.5cm} m{5cm} m{6cm} }
  \textbf{Name} & \textbf{Symbol} & \textbf{$:=$ Definition} & \textbf{in Worten/Kommentar} \\ 
  \hline
  $\sigma$-Algebra& $\fa$ & $\fa\subseteq\mathcal{P}(X)$ mit 
	\begin{enumerate}
	\item[($\sigma_1$)] $X\in\fa$
	\item[($\sigma_2$)] $A\in\fa ⇒ A^c\in\fa$.
	\item[($\sigma_3$)] $(A_j)$ ist Folge in $\fa $ \newline $⇒ \bigcup A_j\in\fa$.
	\end{enumerate} & Das sind alle mölicher Kombinationen unter Veneinigung, Komplement, Schnitt, etc. \\
  \hline
  Erzeuger & $\sigma(\mathcal{E})$ & $$\bigcap_{\fa\in\cf}\fa $$ \newline mit $\cf :=$ Menge der $\sigma$ Algebren auf $X$, die $\mathcal{E}$ enthalten  & Kleintse $\sigma$-Algebra, die  $\mathcal{E}$ enthält. \newline „Die von $\mathcal{E}$ erzeugte $\sigma$-Algebra“ \\
  \hline
  & $\mathcal{O}(X)$ & $\{A\subseteq X:A$ offen in $X\}$ & Menge der offenen Mengen, die $X$ enthalten \\
  \hline
  Borelsche \newline $\sigma$-Algebra & $\fb(X)$ & $\sigma(\mathcal{O}(X))$ & „Borelsche $\sigma$-Algebra auf $X$“ \\
  \hline
  & $\fb_d$ & $\fb(\mdr^d)$ & Elemente von $\fb_d$ heißen \textbf{Borelsche Mengen} \\
  \hline
  Intervall in $\mdr^d$& $ $ & $I_1,\ldots,I_d$ Intervalle in $\mdr$ \newline 
  $⇒ I_1\times\cdots\times I_d$ Intervall in $\mdr^d$ & Quader senkrecht zu den Achsen \\
  \hline
  & $a\le b$ & $ a_j\le b_j\quad (j=1,\ldots,d)$ & Vergleich von Intervallen ist Komponentenweise definiert \\
  \hline
  offene \newline Intervalle in $\mdr^d$& $ $ & $ $ & analog zu $\mdr$ \\
  \hline
  geschlossene Intervalle in $\mdr^d$& $ $ & $ $ & analog zu $\mdr$ \\
  \hline
  Halbraum & $H_k^-(\alpha)$ & $\{(x_1,\ldots,x_d)\in\mdr^d:x_k\le\alpha\}$ & alles, was in der $k$-ten Dimension kleiner (oder gleich) ist als $α$ \\
  \hline
  & $H_k^+(\alpha)$ & $ $ & analog \\
  \hline
  Spur & $\fm_Y$ & $\{A \cap Y : A \in \fm\}$ \newline 
  mit $\varnothing \neq \fm \subseteq \mathcal{P}(X)$ \newline und $\varnothing \neq Y \subseteq X$ 
  & „Spur von $\fm$ in $Y$“ \\
  \hline
  &$+\infty$& & Definition trivial und intuitiv, aber lang. Daher hier nicht aufgeführt. \\
  \hline
  Maß&&
\begin{enumerate}
\item[$(M_1)$] $\mu(\varnothing)=0$
\item[$(M_2)$] $(A_j)$ disjunkte Folge in $\fa$ \newline $⇒ \mu(\bigcup A_j)=\sum\mu(A_j)$
\end{enumerate}&\\
  \hline
  $σ$-Additivität&& Anderer Name für $(M2)$&\\
  \hline
  Maßraum &$(X,\fa,\mu)$& $\fa$ ist $\sigma$-Algebra auf $X$ \newline und $\mu$ ist Maß auf $\fa$&\\
  \hline
  endlichen Maß&&$\mu(X)<\infty$& Reminder: $X$ ist die Basismenge\\
  \hline
  Wahrscheinlichkeitsmaß&&$\mu(X)=1$&\\
  \hline
	Punktmaß\newline / Dirac-Maß &  $\delta_{x_0}(A)$ & $\begin{cases}
	1,\ x_0\in A\\
	0,\ x_0\not\in A
	\end{cases}$ & \emph{misst}, ob der \emph{Punkt} in der Menge ist \\
	\hline
  &&&\\
  \hline
  Zählmaß&$\mu:\fa\to[0,+\infty]$& 
	$\mu(A):=\begin{cases}
	0&,A=\varnothing\\
	\sum_{j\in A}p_j&,A\ne\varnothing
	\end{cases}$
	\newline mit $(p_j)$ Folge in $[0,+\infty]$&Summiert/Zählt die Folgenglieder, deren index in der Menge A sind \\
  
\end{longtable}


\chapter{Das Lebesguema\ss}
\label{Kapitel 2}
\index{Lebesguemaß}

Ziel dieses Kapitels: Fortsetzung von \(\lambda_{d}\) auf \(\cf_{d}\) und dann auf \(\fb_d\) (\(\leadsto\) Lebesguema\ss)

Beachte: \(\ci_{d}\subseteq\cf_{d}\subseteq\fb_{d}\overset{1.4}{\implies}\fb_{d}=\sigma(\ci_{d})=\sigma(\cf_{d})\)

\textbf{Definitionen}:

\begin{longtable}{ m{2.5cm} m{3cm} m{5cm} m{6cm} }
  \textbf{Name} & \textbf{Symbol} & \textbf{$:=$ Definition} & \textbf{in Worten/Kommentar} \\ 
  \hline
	Ring  & $\varnothing \neq \mathfrak{R} \subseteq \mathcal{P}(X) $ & 
	\begin{enumerate}
	\item \(\varnothing\in\mathfrak{R}\)
	\item \(A,B\in\mathfrak{R} \implies A\cup B,\,B\setminus A\in\mathfrak{R}\)
	\end{enumerate} & Mengensystem, das unter Vereinigung und Differenz abg. ist\\
  \hline
  Elementarvo- lumen
  	& $\lambda_{d}: \ci_{d} → [0,\infty)$ 
	& $\begin{cases}0&\text{falls }I=\varnothing\\(
	  \sum_{i\le d} (b_{i}-a_{i})&\text{falls }I\neq\varnothing
	  \end{cases}$ mit $a,b \in \mdr^d, I=(a,b] \in \ci_{d}$ 
	& Normales Volumen in $\mdr$ \\
  \hline
  Menge der Figuren
  	& $\cf_d$ 
  	& $$\left\{\bigcup_{I_{j}\in I} I_{j} \right\} \text{ mit } I \subseteq \ci_{d}, |I| \in \mdn$$ 
  	& \\
  \hline
  	& $\lambda_{d}:\cf_{d}\to [0,\infty)$ 
  	& $$\sum_{j=1}^{n}{\lambda_{d}(I_{j})}$$
  		\newline mit $\{I_{1},\ldots,I_{n}\}\subseteq\ci_{d}$ 
  		\newline und  $I_j$ disjunkt
  	& Fortsetzung von $\lambda_{d}$ von $\ci_{d}$ nach $\cf_d$   \\
  \hline Prämaß
  	& $\mu:\fr\to[0,\infty]$ 
  	& 
		\begin{enumerate}
		\item \(\mu(\varnothing)=0\)
		\item \(\mu\left(\bigcup{A_{j}}\right)=\sum{\mu(A_{j})}\) 
			\newline wenn $A_{j}$ disj. Folge in \(\fr\)
			\newline und \(\bigcup{A_{j}}\in\fr\)
		\end{enumerate}
  	& \(\lambda_{d}:\cf_{d}\to[0,\infty]\) ist ein Pr\"ama\ss .\\
  \hline \textbf{Lebesguemaß} \newline (auch L-Ma\ss)
  	& $\lambda_{d}:\fb_{d}\to[0,\infty]$ 
  	& $ $ 
  	& Fortsetzung von \(\lambda_{d}:\cf_{d}\to[0,\infty]\) auf
\(\fb_{d}\). \newline ist Maß \newline ist eindeutig\\
  \hline
  	& $x+B$ 
  	& $$x+B:=\{x+b\mid b\in B\}$$ mit $x\in\mdr^d, B\subseteq\mdr^d$
  	& \\
  	
  \hline Auswahlaxiom
  	& 
  	\multicolumn{3}{m{12cm}}{Sei $\varnothing\ne\Omega$ Indexmenge, es sei $\{X_\omega\mid \omega\in\Omega\}$ ein disjunktes System von nichtleeren Mengen $X_\omega$. Dann existiert ein $C\subseteq\bigcup_{\omega\in\Omega}X_\omega$, sodass $C$ mit jedem $X_j$ genau ein Element gemeinsam hat.
  	}\\
  	\\
\end{longtable}

\chapter{Messbare Funktionen}
\label{Kapitel 3}


Definitionen:

\begin{longtable}{ m{2.5cm} m{3cm} m{5cm} m{6cm} }
  \textbf{Name} & Symbol & Definition & in Worten/Kommentar \\ 
  \hline messbarer Raum
  	& $(X,\fa)$ 
  	& wenn $\fa$ eine $\sigma$-Algebra auf $X$ ist
  	& \\
  \hline  $\fa$-$\fb$-messbar
  	& Funktion $f$ heißt \textbf{$\fa$-$\fb$-messbar} 
  	& $$\forall B\in\fb: f^{-1}(B)\in\fa$$ 
  		 mit  $\fa$ eine $\sigma$-Algebra auf $X$
  		\newline und $\fb$ eine $\sigma$-Algebra auf $Y$ 
  		\newline und $f:X\to Y$
  	& \\
  \hline messbare Funktion
  	& $ $ 
  	& i.d.R.äquiv. zu Borelmessbar 
  	& \\
  \hline messbare Funktion (alternatve Def.)
  	& $f: X ⇒ \mdr$ ist messbar 
  	& $ ∃ (f_n)$ mit $f_n:
  	X⇒\mdr$ und $\lim_{n↦\inf} f_n(x) =f(x) $
  	\newline
  	& Diese Def. wird in weniger theoretischen Vorlesungen verwendet. Diese Def. ist gleichwertig zur obigen Def. \\
  \hline Borel-messbare Funktion
  	& $ $ 
  	& $\fb(X)-\fb_{k}$-messbar 
  		\newline mit $X\in\fb_{d}$
  		\newline und $f:\,X\to\mdr^{k}$
  	& Erinnerung: \newline \(\fb(X)=\{A\in\fb_{d}\mid A\subseteq X\}\) \\
  	
  \hline
  &$-\infty$& & analog $+\infty$ (siehe letztes Kapitel) \\
  \hline
  	&$\imdr$
  	& $:=[-\infty,+\infty]$\newline$:=\mdr\cup\{-\infty,+\infty\}$
  	& \\
  \hline Borelsche \(\sigma\)-Algebra auf $\imdr$.
  	& $\ifb_{1}$ 
  	& $\{B\cup E\mid B\in\fb_{1},\,E\subseteq\{-\infty,+\infty\}\}$ 
  	& „Borelmessbar“ kann sich auch auf dieses hier beziehen \\
  	
  \hline 
  	& $-M$ 
  	& $\{-m\mid m\in M\}$ 
  	& \\
  \hline  Supremum
  	& $\sup M$ \newline mit $M\subseteq\imdr$ 
  	& Definition trivial und intuitiv, aber lang. Daher hier nicht aufgeführt.  
  	& Kleinste obere Grenze  (u.U. von einer unbeschräkten Menge) \\
  \hline  Inferum
  	& $\inf M$ 
  	& $-\sup(-M)$ 
  	& Größte untere Grenze (u.U. von einer unbeschräkten Menge)\\
  \hline 
  	& $\sup_{n\in\mdn}(f_n):$\newline$X\to\imdr$
  	& $\sup\{f_n(x)\mid n\in\mdn\}\quad x\in X$ 
  	& \\
  \hline 
  	& $\inf_{n\in\mdn}(f_n):$\newline$ X\to\imdr$ 
  	& $\inf\{f_n(x)\mid n\in\mdn\}\quad x\in X$ 
  	& \\
  \hline 
  	& $\limsup_{n\to\infty} f_n:$\newline$X\to\imdr$ 
  	& $\inf_{j\in\mdn}(\sup_{n\ge j} f_n)$ 
  	& \\
  \hline 
  	& $\liminf_{n\to\infty} f_n:$\newline$X\to\imdr$ 
  	& $\sup_{j\in\mdn}(\inf_{n\ge j} f_n)$ 
  	& \\
  \hline 
  	& $\limsup_{n\to\infty} a_n$ \newline ($a_n$ beschränkte Folge in $\mdr$) 
  	& $\inf\{\sup\{a_n\mid n\ge j\}\mid j\in\mdn\}$ 
  	& Erinnerung aus Ana 2\\
  \hline
  	& $\max_{1\le n\le N} f_n$
  	& $\sup_{j\in\mdn} g_n$ 
  	\newline \tiny mit $N\in\mdn$ fest
  	\newline und $g_j:=f_j$ (für $j=1,\ldots,N$), 
  	\newline und $g_j:=f_N$ (für $j>N$)
  	& \\
  \hline 
  	& $\min_{1\le n\le N} f_n$ 
   	& $\inf_{j\in\mdn} g_n$ 
   		\newline \tiny $N,j, g_j$ wie oben
  	& \\
  \hline 
  	&  $\lim_{n\to\infty} f_n:$\newline$  X\to\imdr$\newline \tiny wenn $f_n(x)$  für jedes $x\in\imdr$ konvergent
  	& $ $ 
  	& \\
  \hline Positivteil
  	& $f_+$ 
  	& $\max\{f,0\}$ 
  	& \\
  \hline Negativteil
  	& $f_-$ 
  	& $\max\{-f,0\}$ 
  	& \\
  \hline Treppenfunktion \newline Einfache Funktion 
  	& 
  	& $f:X\to\mdr$ ist messbar 
  	  \newline und $f(X)$ ist endlich 
  	& \\
  \hline Normalform
  	& Normalform von $f$ 
  	& $f=\sum_{j=1}^m y_j \mathds{1}_{A_j}$
  		\tiny \newline mit $f$ einfach
  		\newline $f(X)=\{y_1,\ldots,y_m\}$ mit $y_i\ne y_j$ für $i\ne j$
  		\newline $A_j:=f^{-1}(\{y_j\})$ für $j=1,\ldots,m$ 
  	& Außerdem: $A_1,\ldots,A_m\in\fb(X)$ und $X=\bigcup_{j=1}^m A_j$ disjunkte Vereinigung\\
  \hline zulässig
  	& $(f_n)$ heißt zulässig für $f$
  	&  $f_n(x)\stackrel{n\to\infty}{\to}f(x)$ ($\forall x\in X$)
  		\tiny \newline mit $0\le f_n\le f_{n+1}$ auf $X$ ($\forall n\in\mdn$) 
  		\newline $f:X\to\imdr$ messbar
  		\newline $f\ge 0$ auf $X$
  		\newline $(f_n)$ Folge einfacher Funktionen $f_n:X\to[0,\infty)$
  	& „zulässig“ ist keine offizielle Definition und wird nur von Hr. Schnöger benutzt \\
\end{longtable}


\chapter{Konstruktion des Lebesgueintegrals}
\label{Kapitel 4}


Definitionen:

\begin{longtable}{ m{2.5cm} m{3cm} m{5cm} m{6cm} }
  \textbf{Name} & Symbol & Definition & in Worten/Kommentar \\ 
  \hline Lebesgueintegral \newline (Funktion)
  	& $$\int_X f(x)\text{ d}x$$ 
  	& $$\sum_{j=1}^m y_j\lambda(A_j)$$ \newline \tiny mit $f:X\to [0,\infty)$
  	& \\
  \hline  Lebesgueintegral \newline (Funktionsfolge)
  	& $$\int_X f(x)\text{ d}x$$ 
  	& $$\lim_{n\to\infty}\int_X f_n(x)\text{ d}x$$ \newline \tiny mit $f:X\to[0,\infty]$ messbar \newline und $(f_n)$ eine für $f$ zulässige Folge 
  	& \\
  \hline 
  	& $M(f)$ 
  	& $$\{\int_X g\text{ d}x\mid g:X\to[0,\infty) \text{ einfach und }g\le f\text{ auf }X\}$$\newline \tiny \newline  mit $f:X\to[0,\infty]$ messbar
  	& \\
  \hline 
  	& $L$ 
  	& $$\lim_{n\to\infty}\int_X f_n\text{ d}x=\sup M(f)$$ \tiny 
  	\newline mit $f:X\to[0,\infty]$ messbar
  	\newline und $(f_n)$ zulässig für $f$
  	& \\
  \hline (Lebesgue-) \textbf{Integral} von $f$ (über $X$)
  	& $$\int_X f(x) \text{ d}x$$
  	& $$\int_X f_+(x) \text{ d}x-\int_X f_-(x) \text{ d}x$$
  		\tiny \newline mit $f:X\to\imdr$ messbar
  		\newline und $\int_X f_+(x) \text{ d}x<\infty$
  		\newline und $\int_X f_-(x) \text{ d}x<\infty$
  	& \\
  \hline integrierbar
  	& $ $ 
  	& $$\int_X f(x) \text{ d}x<\infty$$ 
  		\newline mit $f:X\to[0,\infty]$ messbar
  	& \\
  \hline 
  	& $\fl^1(X)$ 
  	& $\{f: X \to \mdr \mid f$ ist messbar und $\int_X \lvert f \rvert \text{ d}x < \infty\}$ 
  	& \\
\end{longtable}


\chapter{Nullmengen}
\label{Kapitel 5}


Definitionen:

\begin{longtable}{ m{2.5cm} m{3cm} m{5cm} m{6cm} }
  \textbf{Name} & Symbol & Definition & in Worten/Kommentar \\ 
  \hline (Borel-) Nullmenge
  	& $N$ ist Nullmenge
  	& $\lambda(N)=0$ 
  	\newline und $N\in\fb_d$
  	& Menge, deren integral Null ist \\
  \hline für fast alle / \newline fast überall
  	& [Eigenschaft] (E) gilt ffa $x\in X$ 
  	& $∃ N\subseteq X$ sodass $(E)$ für alle $x\in X\setminus N$ gilt. 
  	& kurz „ffa“ \ „fü“ \\
  \hline Konvergenz ffa
  	& Folge $(f_n)$ konvergiert fast überall (auf $X$)  
  	& Nullmenge $N\subseteq X$ existiert, sodass für alle  $x\in X\setminus N$ $\left(f_n(x)\right)$ in $\imdr$ konvergiert.
  	& \\
  \hline Konvergenz ffa gegen 
  	& $f_n\to f$ fast überall \tiny \newline für $f:X\to\imdr$
  	& Nullmenge $N\subseteq X$ existiert mit: $f_n(x)\to f(x) \forall x\in X\setminus N$
  	& „$(f_n)$ konvergiert fast überall (auf $X$) gegen $f$“ \\
\end{longtable}


\chapter{Der Konvergenzsatz von Lebesgue}
\label{Kapitel 6}



Definitionen:

\begin{longtable}{ m{2.5cm} m{3cm} m{5cm} m{6cm} }
  \textbf{Name} & Symbol & Definition & in Worten/Kommentar \\ 
  \hline Majorisierte Konvergenz
  	& $ $ 
  	& siehe Ana 1 
  	& \\
  \hline Majorante
  	& $ $
  	& siehe Ana 1 
  	& \\
  \hline 
  	& $ $ 
  	& $ $ 
  	& \\
\end{longtable}



\chapter{Parameterintegrale}
\label{Kapitel 7}


Definitionen:

\begin{longtable}{ m{2.5cm} m{3cm} m{5cm} m{6cm} }
  \textbf{Name} & Symbol & Definition & in Worten/Kommentar \\ 
  \hline Intergal 
  	& $F(t)$ 
  	& $$\int_Xf(t,x)\,dx$$ 
  		\tiny \newline mit $f\colon U\times X\to \mdr$ integrierbar
  		\newline Es existiert eine Nullmenge \(N\subseteq X\) so, dass \(t\mapsto f(t,x)\) für jedes \(x\in X\setminus N\) stetig in $t_0$ ist.
  	& \\
  \hline integrierbar
  	&  $x\mapsto f(t,x)$ ist für jedes $t\in U$ integrierbar
  	& \tiny  $U\in\fb_k, t_0\in U$ 
  		\newline $(1)$ $∀ t\in U:x\mapsto f(t,x)$ messbar
  		\newline $(2)$ $∃$ Nullmenge $N\subseteq X: ∀ x\in X\setminus N: t\mapsto f(t,x)$stetig in $t_0$
		\newline $(3)$ $∃$ integrierbare Funktion \(g\colon X\to [0,\infty]\) und $ ∀t\in U: ∃$ Nullmenge \(N_t\subseteq X:$\newline$ ∀ t\in U, x\in X\setminus N_t\) gilt: $ \lvert f(t,x)\rvert \leq g(x) $
  	& \\
  \hline partielle Integration
  	& $$\frac{\partial F}{\partial t_j}(t)$$
  	& $$ \int_X\frac{\partial f}{\partial t_j}(t,x)\,dx $$
	  	\tiny \newline mit
	  	\newline (1) $f(t,x)$ part. differenzierbar
	  	\newline (2) $f(x,t)$ integriebar
	  	\newline (3) $\left\lvert \frac{ \partial f}{\partial t_j} \right\rvert \leq g(x)$, $∀x \in X \setminus N ,j \in \mdn$
	  	
  	& keine Ahnung, wo das $g$ herkommt\\
\end{longtable}


\chapter{Vorbereitungen auf das, was kommen mag}
\label{Kapitel 8}


Definitionen:

\begin{longtable}{ m{2.5cm} m{3cm} m{5cm} m{6cm} }
  \textbf{Name} & Symbol & Definition & in Worten/Kommentar \\ 
  \hline 
  	& $p_1\colon\mdr^d\to\mdr^k$ 
  	& $p_1(x,y):=x$ 
  	& \\
  \hline 
  	& $ p_2\colon\mdr^d\to\mdr^l$ 
  	& $p_2(x,y):=y$ 
  	& \\
  \hline 
  	& $j_y\colon\mdr^k\to\mdr^d$ 
  	& $j_y(x):=(x,y)$
  		\newline \tiny mit $\in\mdr^l$
  	& \\
  \hline 
  	& $j^x\colon\mdr^l\to\mdr^d$ 
  	& $j^x(y):=(x,y)$
  		\newline \tiny mit $x\in\mdr^k$
  	& \\
  \hline y-Schnitt
  	& $C_y$ 
  	& $\{x\in\mdr^k:(x,y)\in C\}=(j_y)^{-1}(C)$ 
  		\newline \tiny mit $C\subseteq\mdr^d$
  	& \\
  \hline x-Schnitt
  	& $C^x$ 
  	& $\{y\in\mdr^l:(x,y)\in C\}=(j^x)^{-1}(C)$  
  		\newline \tiny mit $C\subseteq\mdr^d$
  	& \\
  \hline 
  	& $f_y(x)$ 
  		\newline \tiny mit $f\colon\mdr^d\to\imdr$
  		\newline $y$ fest
  	& $f(x,y) \ \ (x\in\mdr^k)$ 
  	& $f_y=f\circ j_y$\\
  \hline 
  	& $f^x(y)$ 
  		\newline \tiny mit $f\colon\mdr^d\to\imdr$
  		\newline $x$ fest
  	& $f(x,y) \ \ (y\in\mdr^l)$ 
  	& $f^x=f\circ j^x$\\
  \hline 
  	& $\varphi_C(x)$ 
  		\newline \tiny mit $C\in\fb_d$
  	& $\lambda_l(C^x) \ \ (x\in\mdr^k) $ 
  	& \\
  \hline 
  	& $\psi_C(x)$ 
  	& $\lambda_k(C_y) \ \ (y\in\mdr^l) $ 
  	& \\
\end{longtable}



\chapter{Das Prinzip von Cavalieri}
\label{Kapitel 9}


\begin{satz}[Prinzip von Cavalieri]
\label{Satz 9.1}
Sei \(C\in\fb_d\). Dann:
\[ \lambda_d(C)=\int_{\mdr^k}\lambda_l(C^x)\,dx=\int_{\mdr^l}\lambda_k(C_y)\,dy \]
\end{satz}
Das heißt:
\[ \int_{\mdr^d}\mathds{1}_{C}(x,y) \text{ d}(x,y) = \int_{\mdr^k}\left(\int_{\mdr^l} \mathds{1}_{C}(x,y)\,dy\right)\,dx = \int_{\mdr^l} \left(\int_{\mdr^k} \mathds{1}_{C}(x,y)\,dx\right)\,dy \]


Definitionen:

\begin{longtable}{ m{2.5cm} m{3cm} m{5cm} m{6cm} }
  \textbf{Name} & Symbol & Definition & in Worten/Kommentar \\ 
  \hline 
  	& $\tilde f:\mdr^d\to\imdr$ 
  	 \newline \tiny mit $\varnothing\ne D\in\fb_d$ 
  	 \newline und $f:D\to\imdr$ messbar
  	& $f(z) := \begin{cases} f(z) &,z\in D\\ 0&,z\not\in D\end{cases}$ 
  	& $\tilde f$ messbar\\
  \hline 
  	& $ $ 
  	& $ $ 
  	& \\
\end{longtable}

\chapter{Der Satz von Fubini}
\label{Kapitel 10}


Definitionen:

\begin{longtable}{ m{2.5cm} m{3cm} m{5cm} m{6cm} }
  \textbf{Name} & Symbol & Definition & in Worten/Kommentar \\ 
  \hline iterierte Integrale
  	& \multicolumn{3}{m{12cm}}{$$\int_{\mdr^d}f(x,y)\,d(x,y)=\int_{\mdr^k}\left(\int_{\mdr^l}f(x,y)\,dy\right)dx=\int_{\mdr^l}\left(\int_{\mdr^k}f(x,y)\,dx\right)dy$$}\\
\end{longtable}


\begin{satz}[Satz von Fubini (Version II)]
\label{Satz 10.3}
Sei \(\varnothing\neq X\in\fb_k\), \(\varnothing\neq Y\in\fb_l\) und \(D:=X\times Y\) (nach \S 8 ist \(D\in\fb_d\)).
Es sei \(f\colon D\to\imdr\) messbar.
Ist \(f\geq 0\) auf $D$ oder ist $f$ integrierbar, so gilt
\[ \int_D f(x,y)\,d(x,y) = \int_X\left(\int_Yf(x,y)\,dy\right)dx = \int_Y\left(\int_Xf(x,y)\,dx\right)dy \]
\end{satz}

\textbf{"'Gebrauchsanweisung"' für Fubini:}\\
Gegeben: \(\varnothing\neq D\subseteq\fb_d\) und messbares \(f\colon D\to\imdr\).
Setze $f$ auf \(\mdr^d\) zu einer messbaren Funktion \(\tilde f\) fort (zum Beispiel wie in \ref{Lemma 9.3}).
Aus \ref{Satz 3.8} folgt dann, dass \(\mathds{1}_{D}\tilde f\) messbar ist und \ref{Satz 10.1} liefert
	\begin{align*}
	\int_{\mdr^d}\lvert \mathds{1}_{D}\tilde f\rvert\,dz 
	= \int_{\mdr^k}\left(\int_{\mdr^l}\lvert \mathds{1}_{D}\tilde f\rvert\,dy\right)dx 
	= \int_{\mdr^l}\left(\int_{\mdr^k}\lvert \mathds{1}_{D}\tilde f\rvert\,dx\right)dy
	\end{align*}
Ist eines der drei obigen Integrale endlich, so ist \(\lvert \mathds{1}_{D}\tilde f\rvert\) integrierbar und
damit ist nach \ref{Satz 4.9} auch \(\mathds{1}_{D}\tilde f\) integrierbar.\\
Dann ist $f$ integrierbar und es folgt
	\begin{align*}
	\int_Df(z)\,dz 
	& = \int_{\mdr^d}\left(\mathds{1}_{D}\tilde f\right)(z)\,dz									\\
	& \overset{\ref{Satz 10.2}}= \int_{\mdr^k}\left(\int_{\mdr^l}\left(\mathds{1}_{D}\tilde f\right)(x,y)\,dy\right)dx 	\\
	& = \int_{\mdr^l}\left(\int_{\mdr^k}\left(\mathds{1}_{D}\tilde f\right)(x,y)\,dx\right)dy
	\end{align*}




\chapter{Der Transformationssatz (Substitutionsregel)}
\label{Kapitel 11}


Definitionen:

\begin{longtable}{ m{2.5cm} m{3cm} m{5cm} m{6cm} }
  \textbf{Name} & Symbol & Definition & in Worten/Kommentar \\ 
  \hline 
  	& $ $ 
  	& $ $ 
  	& \\
\end{longtable}

Die Sätze in diesem Paragraphen geben wir \textbf{ohne} Beweis an. Es seien
\(X,Y\subseteq\mdr^d\) nichtleer und offen. 

\begin{definition}
\index{Diffeomorphismus}
Sei \(\Phi\colon X\to Y\) eine Abbildung. \(\Phi\) heißt 
\textbf{Diffeomorphismus} genau dann wenn \(\Phi\in C^1(X,\mdr^d)\), \(\Phi\)
ist bijektiv und \(\Phi^{-1}\in C^{1}(Y,\mdr^d)\).\\
Es gilt \[x=\Phi^{-1}(\Phi(x))\text{ für jedes } x\in X\]
Kettenregel: \[I=\left(\Phi^{-1}\right)^\prime(\Phi(x))\cdot\Phi^\prime(x)
\text{ für jedes } x\in X\] Das heißt \(\Phi^\prime(x)\) ist invertierbar für
alle \(x\in X\) und somit ist \(\det\left(\Phi^\prime(x)\right)\neq 0\)
für alle \(x\in X\).
\end{definition}

\begin{satz}[Transformationssatz (Version I)]
\label{Satz 11.1}
\(\Phi\colon X\to Y\) sei ein Diffeomorphismus.
\begin{enumerate}
\item	\(f\colon Y\to[0,+\infty]\) sei messbar und für \(x\in X\) sei
	\(g(x):=f\left(\Phi(x)\right)\cdot\lvert\det\Phi^\prime(x)\rvert\).\\
	Dann ist \(g\) messbar und es gilt:
	\begin{align*}\tag{$*$} \int_Yf(y)\,dy=\int_Xg(x)\,dx=\int_Xf\left(\Phi(x)\right)
	\cdot\lvert\det\Phi^\prime(x)\rvert\,dx\end{align*}
\item	\(f\colon Y\to\imdr\) sei integrierbar und $g$ sei definiert wie in (1).
	Dann ist $g$ integrierbar und es gilt die Formel \((\ast)\).
\end{enumerate}
\end{satz}

\begin{erinnerung}
\index{Inneres}
Sei \(A\subseteq\mdr^d\) und \(A^\circ:=\{x\in A :\text{ es existiert ein } r=r(x)>0
\text{ mit } U_r(x)\subseteq A\}\) das \textbf{Innere} von $A$. $A^\circ$ ist offen!
\end{erinnerung}

\begin{beispiel}
Sei \(A=\mdr\setminus\mdq\). Es ist \(A^\circ=\varnothing\) und 
\(A\setminus A^\circ=A\). Aus \(\mdr=A\dot\cup\mdq\) folgt
\[\infty=\lambda_1(\mdr)=\lambda_1(A)+\lambda_1(\mdq)=\lambda_1(A)\]
Das heißt \(A\setminus A^\circ\) ist keine Nullmenge.
\end{beispiel}


\begin{satz}[Transformationssatz (Version II)]
\label{Satz 11.2}
Es sei $\varnothing \neq U \subseteq \MdR^d$ offen, $\Phi \in C^1(U, \MdR^d)$, $A \subseteq U$, $A \in \fb_d$,
$X := A^{\circ}$ und $A \setminus A^{\circ}$ eine Nullmenge.
Weiter sei $\Phi$ injektiv auf $X$, $\det\Phi' \neq 0$ für alle $x \in X$, $B:=\Phi(A) \in \fb_d$ und
$g(x) = f(\Phi(x)) \cdot \lvert\det\Phi'(x)\rvert$ für $x \in A$.
%% BILD: von Phi und Mengen
Dann gilt:
\begin{enumerate}
\item	$Y := \Phi(X)$ ist offen und $\Phi: X\to Y$ ist ein Diffeomorphismus.
\item	Ist $f\colon B \to [0, \infty]$ messbar, so ist $g\colon A \to [0, \infty]$ messbar und
\[ \int_B f(y) \, dy = \int_A g(x) \, dx= \int_A f(\Phi(x)) \cdot\lvert\det(\Phi'(x))\rvert \, dx \qquad (\ast\ast)\]
\item	Ist $f\colon B \to \imdr$ messbar, so gilt:\\
\[ f \in \fl^{1}(B) \gdw g \in \fl^{1}(A) \]
Ist $f \in \fl^{1}(B)$ so gilt $(\ast\ast)$
\end{enumerate}
\end{satz}

\begin{folgerungen}
\label{Folgerung 11.3}
\begin{enumerate}
\item	Sei $T\colon \MdR^d \to \MdR^d$ linear und $\det T \neq 0$. Weiter sei $A \in \fb_d$ und $v \in \MdR^d$.
Dann ist $T(A) \in \fb_d$ und es gilt:
\[\lambda_d(T(A)+v) = \lvert\det T\rvert \cdot\lambda_d(A)\]
\item	$\Phi\colon X \to Y$ sei ein Diffeomorphismus und $A \in \fb(X)$. 
Dann ist $\Phi(A) \in \fb_d$ und es gilt:
\[\lambda_d(\Phi(A)) = \int_A |\det \Phi'(X)| \, dx\]
\item	Sei $F \in C^1(X, \MdR^d)$ und $N \subseteq X$ eine Nullmenge.
Dann ist $F(N)$ enthalten in einer Nullmenge.
\end{enumerate}
\end{folgerungen}

\begin{beispiel}
Seien $a,b > 0$ und $T:=\begin{pmatrix} a &  0 \\ 0 & b \end{pmatrix}$, $\det T = a b > 0$. Definiere:
\[A:=\{(x,y)\in \MdR^2: x^2 + y^2 \leq 1\}\]
Dann ist $A \in \fb_2$ und $\lambda_2(A) = \pi$.
\begin{align*}
(u,v) \in T(A) &\gdw \exists (x,y)\in A: (u,v) = (a x, b y)\\
&\gdw \exists (x,y) \in A: (x = \frac{u}{a})\wedge (y = \frac{v}{b})\\
&\gdw \frac{u^2}{a^2} + \frac{v^2}{b^2} \leq 1
\end{align*}
%% BILD: einer Ellipse
Aus \ref{Folgerung 11.3} folgt $T(A) \in \fb_2$ und $\lambda(T(A)) = a b \pi$.
\end{beispiel}

\setcounter{section}{3}
\section{Polarkoordinaten}
\index{Polarkoordinaten}
%% BILD: von PK neben Formeln
%% Tabellarisches Layout?
Jeder Vektor im $\mdr^2$ lässt sich nicht nur durch seine Projektionen auf die Koordinatenachsen $(x,y)$, sondern auch eindeutig durch seine Länge $r$ und den (kleinsten positiven) Winkel $\varphi$ zur $x$-Achse darstellen. Diese Darstellung $(r,\varphi)$ heißen die \textbf{Polarkoordinaten} des Vektors. Dabei gilt:
\[r = \|(x,y)\| = \sqrt{x^2 + y^2}\]
und
\[\begin{cases}
x = r \cos(\varphi)\\
y = r \sin(\varphi)
\end{cases}\]
Definiere nun für $(r,\varphi) \in [0,\infty)\times[0,2\pi]$:
\[\Phi(r,\varphi) := (r \cos(\varphi), r \sin(\varphi))\]
Dann ist $\Phi \in C^1(\MdR^2, \MdR^2)$ und es gilt: 
\[\Phi'(r,\varphi) = \begin{pmatrix}
\cos(\varphi) & -r \sin(\varphi) \\ 
\sin(\varphi) & r \cos(\varphi)
\end{pmatrix}\]
d.h. falls $r > 0$ ist gilt:
\[\det\Phi'(r,\varphi) = r \cos^2(\varphi) + r \sin^2(\varphi) = r > 0\]


\begin{bemerkung}[Faustregel für Polarkoordinaten]
Ist ein Integral der Form $\int_B f(x,y) d(x,y)$ zu berechnen, so lässt sich oft eine Menge $A$ finden, sodass $\Phi(A) = B$ ist.
%% BILD: Kreissektor <=> Rechteck
Mit \ref{Satz 11.2} folgt dann:
\[\int_B f(x,y) \text{ d}(x,y) = \int_A f(r \cos \varphi, r \sin \varphi) \cdot r \text{ d}(r,\varphi)\]
\end{bemerkung}

\begin{beispiel}
\begin{enumerate}
\item	Sei $0 \le \rho < R$. Definiere 
\[B := \{(x,y) \in \MdR^2 : \rho^2 \le x^2 + y^2 \le R^2\} \]
Dann gilt: 
%% BILD: der Kreisfläche und Trafo
\begin{align*}
\lambda_2(B) &= \int_B 1 \text{ d}(x,y)\\ 
&= \int_A 1 \cdot r \text{ d}(r,\varphi)\\ 
&\overset{\text{§\ref{Kapitel 10}}}= \int_{\rho}^{R} \left( \int_0^{2\pi} r \text{ d}\varphi \right) \text{ d}r\\
&= \left[ 2\pi \frac{1}{2} r^2 \right]_\rho^R\\
&= \pi (R^2 - \rho^2)
\end{align*}
		
\item	Definiere 
\[B := \{ (x,y) \in \MdR^2 : x^2 + y^2 \le 1, y \ge 0 \}\]
%% BILD: der (Halb)Kreisfläche und Trafo
Dann gilt:
\begin{align*}
\int_B y \sqrt{x^2+y^2} \text{ d}(x,y) &= \int_A r \sin(\varphi) r \cdot r \text{ d}(r,\varphi)\\
&= \int_A r^3 \sin\varphi \text{ d}(r,\varphi)	\\
&\overset{\text{§\ref{Kapitel 10}}}= \int_0^\pi \left( \int_0^1 r^3 \sin\varphi \text{ d}r \right) \text{ d}\varphi\\
&= \frac{1}{4} \int_0^\pi \sin\varphi \text{ d}\varphi\\
&= \left[ \frac{1}{4}(-\cos\varphi) \right]_0^\pi\\
&= \frac{1}{4}(1+1) = \frac{1}{2}
\end{align*}
\item	\textbf{Behauptung:} \[\int_{-\infty}^\infty e^{-x^2} \, dx = \sqrt{\pi}\]
\textbf{Beweis:}
%% BILD: Bilder von Kreis und Rechtecktrafos/näherungen
Für $\rho > 0$ sei
\[B_\rho := \{(x,y) \in \MdR^2 \mid x,y\ge 0, x^2+ y^2 \le \rho^2\}\]
Weiterhin sei $Q_\rho := [0,\rho] \times [0,\frac{\pi}2]$ und $f(x,y) = e^{-(x^2 + y^2)}$. Dann gilt:
\begin{align*}
\int_{ B_\rho } f(x,y) \text{ d}(x,y) &= \int_{Q_\rho} e^{-r^2} r\text{ d}(r,\varphi)\\
&\overset{\text{§\ref{Kapitel 10}}}= \int_0^{\frac{\pi}{2}} \left( \int_0^\rho r e^{-r^2} \text{ d}r \right) \text{ d}\varphi	\\
&= \frac{\pi}{2} \left[ -\frac{1}{2} e^{-r^2} \right]_{0}^{\rho}\\
&= \frac{\pi}{2} \left( -\frac{1}{2} e^{-\rho^2} +\frac{1}{2} \right)	\\
& =: h(\rho) \stackrel{\rho \to \infty}\to \frac\pi4
\end{align*}
Außerdem gilt:
\begin{align*}
\int_{Q_\rho} f(x,y) \text{ d}(x,y) &= \int_{Q_\rho} e^{-x^2} e^{-y^2}\text{ d}(x,y)	\\
&= \int_0^\rho \left( \int_0^\rho e^{-x^2} e^{-y^2} \text{ d}y \right) \text{ d}x \\
&= \left( \int_0^\rho e^{-x^2} \text{ d}x \right)^2
\end{align*}
		
Wegen $ B_\rho \subseteq Q_\rho \subseteq B_{\sqrt{2} \rho} $ und $f \ge 0$ folgt:
\begin{center}
\begin{tabular}{cccccc}
&$\int_{B_\rho} f \text{ d}(x,y)$ &$\le$ &$\int_{Q_\rho} f \text{ d}(x,y)$ &$\le$ &$\int_{B_{\sqrt{2} \rho}} f \text{ d}(x,y)$\\
$\implies$ &$h(\rho)$ &$\le$ &$\int_{Q_\rho} f \text{ d}(x,y)$	&$\le$ &$h(\sqrt{2} \rho)$ \\
$\implies$ &$h(\rho)$ &$\le$ &$\left( \int_0^\rho e^{-x^2} \text{ d}x \right)^2$ &$\le$ &$h(\sqrt{2} \rho)$ \\
$\implies$ &$\sqrt{h(\rho)}$ &$\le$ &$\int_0^\rho e^{-x^2} \text{ d}x$ &$\le$ &$\sqrt{h(\sqrt{2} \rho)}$\\
\end{tabular}
\end{center}
Mit $\rho \to \infty$ folgt daraus 
\[\int_0^\infty e^{-x^2} \text{ d}x = \frac{\sqrt{\pi}}{2}\]
und damit die Behauptung.
\end{enumerate}
\end{beispiel}

\section{Zylinderkoordinaten}
\index{Zylinderkoordinaten}
Definiere für $(r,\varphi,z)\in[0,\infty)\times[0,2\pi]\times\mdr$:
\[\Phi(r,\varphi,z):=(r\cos(\varphi),r\sin(\varphi),z)\]
Dann gilt:
\[|\det\Phi'(r,\varphi,z)|=\left|\det
\begin{pmatrix}
\cos(\varphi)&-r\sin(\varphi)&0\\
\sin(\varphi)&r\cos(\varphi)&0\\
0&0&1\end{pmatrix}\right|=r
\]

\begin{bemerkung}[Faustregel für Zylinderkoordinaten]
Ist ein Integral der Form $\int_B f(x,y,z) d(x,y,z)$ zu berechnen, so lässt sich eine Menge $A$ finden, sodass $\Phi(A) = B$ ist.
Mit \ref{Satz 11.2} folgt dann:
\[\int_B f(x,y,z) \text{ d}(x,y,z) = \int_A f(r \cos \varphi, r \sin \varphi, z) \cdot r \text{ d}(r,\varphi,z)\]
\end{bemerkung}

\begin{beispiel}
Definiere
\[B:=\{(x,y,z)\in\mdr^3\mid x^2+y^2\le 1, x,y\ge 0,z\in[0,1]\}\]
Dann gilt:
\begin{align*}
\int_B z+y\sqrt{x^2+y^2}\text{ d}(x,y,z)&=\int_A(z+r\sin(\varphi)\cdot r)\cdot r\text{ d}(r,\varphi,z)\\
&=\int_A rz+r^3\sin(\varphi)\text{ d}(r,\varphi,z)\\
&=\int_0^1(\int_0^{\frac\pi 2}(\int_0^1 rz+r^3\sin(\varphi)\text{ d}r)\text{ d}\varphi)\text{ d}z\\
&=(\int_0^1 r\text{ d}r)\cdot(\int_0^1 z\text{ d}z)\cdot(\int_0^{\frac\pi 2} \text{ d}\varphi)+ (\int_0^1 r^3\text{ d}r)\cdot(\int_0^{\frac\pi 2} \sin(\varphi)\text{ d}\varphi)\cdot(\int_0^1 \text{ d}z)\\
&= \frac\pi 8+\frac14
\end{align*}
\end{beispiel}

\section{Kugelkoordinaten}
\index{Kugelkoordinaten}
Definiere für $(r,\varphi,\theta)\in [0,\infty)\times[0,2\pi]\times[0,\pi]$:
\[\Phi(r,\varphi,\theta):=(r\cos(\varphi)\sin(\theta),r\sin(\varphi)\sin(\theta),r\cos(\theta))\]
Dann gilt (nachrechnen!):
\[\det\Phi'(r,\varphi,\theta)= -r^2\sin(\theta)\]

\begin{bemerkung}[Faustregel für Kugelkoordinaten]
Ist ein Integral der Form $\int_B f(x,y,z) d(x,y,z)$ zu berechnen, so lässt sich eine Menge $A$ finden, sodass $\Phi(A) = B$ ist.
Mit \ref{Satz 11.2} folgt dann:
\[\int_B f(x,y,z) \text{ d}(x,y,z) = \int_A f(r\cos(\varphi)\sin(\theta),r\sin(\varphi)\sin(\theta),r\cos(\theta)) \cdot r^2\sin(\theta) \text{ d}(r,\varphi,\theta)\]
\end{bemerkung}

\begin{beispiel}
Definiere
\[B:=\{(x,y,z)\in\mdr^3\mid 1\le \|(x,y,z)\|\le 2, x,y,z\ge 0\}\]
Dann gilt:
\begin{align*}
\int_B \frac1{x^2+y^2+z^2}\text{ d}(x,y,z)&=\int_A \frac1{r^2}\cdot r^2\cdot\sin(\theta)\text{ d}(r,\varphi,\theta)\\
&=\int_A \sin(\theta)\text{ d}(r,\varphi,\theta)\\
&=\frac\pi2
\end{align*}
\end{beispiel}

\begin{beispiel}[Zugabe von Herrn Dr. Ullmann]
Wir wollen das Kugelvolumen $\lambda_3(K)$ mit $K:=\{(x,y,z)\in\mdr^3\mid\|(x,y,z)\|\le 1\}$ berechnen. Dann ist $K=\Phi(A)$ mit $A:= [0,1]\times[0,2\pi]\times [0,\pi]$. Und es gilt:
\begin{align*}
\lambda_3(K)&=\int_K 1\text{ d}(x,y,z)\\
&=\int_A r^2\sin(\theta)\text{ d}(r,\varphi,\theta)\\
&=\int_0^1(\int_0^{2\pi}(\int_0^\pi r^2\sin(\theta) \text{ d}\theta)\text{ d}\varphi)\text{ d}r\\
&=(\int_0^1 r^2 \text{ d}r)\cdot(\int_0^{2\pi} \text{ d}\varphi)\cdot(\int_0^\pi \sin(\theta) \text{ d}\theta)\\
&=\frac{4\pi}3
\end{align*}
\end{beispiel}

\iffalse


\chapter{Vorbereitungen für die Integralsätze}
\label{Kapitel 12}


Definitionen:

\begin{longtable}{ m{2.5cm} m{3cm} m{5cm} m{6cm} }
  \textbf{Name} & Symbol & Definition & in Worten/Kommentar \\ 
  \hline 
  	& $ $ 
  	& $ $ 
  	& \\
\end{longtable}

\begin{definition}
\index{Kreuzprodukt}
Seien $a=(a_1,a_2,a_3),b=(b_1,b_2,b_3)\in\mdr^3$. Dann heißt
\[a\times b:=(a_2b_3-a_3b_2,a_3b_1-a_1b_3,a_1b_2-a_2b_1)\]
das \textbf{Kreuzprodukt} von $a$ mit $b$.
Mit $e_1=(1,0,0),e_2=(0,1,0),e_3=(0,0,1)$ gilt formal:
\[a\times b = \det\begin{pmatrix}e_1&e_2&e_3\\a_1&a_2&a_3\\b_1&b_2&b_3\end{pmatrix}=\det\begin{pmatrix}e_1&a_1&b_1\\e_2&a_2&b_2\\e_3&a_3&b_3\end{pmatrix}\]
\end{definition}

\begin{beispiel}
Sei $a=(1,1,2), b=(1,1,0)$, dann gilt:
\[a\times b= \det \begin{pmatrix}e_1&1&1\\e_2&1&1\\e_3&2&0\end{pmatrix}=-2e_1-(-2)e_2+(1-1)e_3=(-2,2,0)\]
\end{beispiel}

\textbf{Regeln zum Kreuzprodukt:}
\begin{enumerate}
\item $b\times a= -a\times b$
\item $a\times a=0$
\item $(\alpha a)\times(\beta b)=\alpha\beta(a\times b)$ für $\alpha,\beta\in\mdr$
\item $a\cdot(a\times b)=b\cdot(a\times b)=0$
\end{enumerate}

\begin{definition}
\index{Divergenz}
Sei $\varnothing\ne D\subseteq\mdr^n$, $D$ offen und $f=(f_1,\ldots,f_n)\in C^1(D,\mdr^n)$. Dann heißt
\[\divv f:=\frac{\partial f_1}{\partial x_1}+\cdots+\frac{\partial f_n}{\partial x_n}\in C(D,\mdr)\]
die \textbf{Divergenz} von $f$.
\end{definition}

\begin{definition}
\index{Rotation}
Sei $\varnothing\ne D\subseteq\mdr^3$, $D$ offen und $F=(P,Q,R)\in C^1(D,\mdr^3)$. Dann heißt:
\[\rot F:=(R_y-Q_z,P_z-R_x,Q_x-P_y)\in C(D,\mdr^3)\]
die \textbf{Rotation} von $F$.
Dabei gilt formal:
\[\rot F=(\frac{\partial}{\partial x},\frac{\partial}{\partial y},\frac{\partial}{\partial z})\times(P,Q,R)\]
\end{definition}

\begin{definition}
\index{Tangentialvektor}
Sei $\gamma:[a,b]\to\mdr^n$ ein Weg. Ist $\gamma$ in $t_0\in[a,b]$ differenzierbar mit $\gamma'(t_0)\ne 0$, so heißt $\gamma'(t_0)\in\mdr^n$ \textbf{Tangentialvektor} von $\gamma$ in $t_0$.
\end{definition}

\chapter{Der Integralsatz von Gauß im $\MdR^2$}
\label{Kapitel 13}


Definitionen:

\begin{longtable}{ m{2.5cm} m{3cm} m{5cm} m{6cm} }
  \textbf{Name} & Symbol & Definition & in Worten/Kommentar \\ 
  \hline 
  	& $ $ 
  	& $ $ 
  	& \\
\end{longtable}

In diesem Paragraphen sei $(x_0,y_0)\in\MdR^2$ (fest), es sei $R:[0,2\pi]\to[0,\infty)$ stetig und stückweise stetig differenzierbar und $R(0) = R(2\pi)$. Weiter sei 
\begin{displaymath}
\gamma(t) := (x_0 + R(t)\cos t,y_0 + R(t)\sin t) \text{   } (t\in[0,2\pi])
\end{displaymath}
Dann ist $\gamma$ ein stückweise stetig differenzierbarer, geschlossener und rektifizierbarer Weg in $\MdR^2$. Es sei 
\[B:= \{(x_0+r\cos t,y_0 + r\sin t): t\in [0,2\pi ], 0\le r\le R(t)\}\] 
Dann ist $B$ kompakt, also $B\in\fb_2 $. Weiter ist $\partial B = \gamma([0,2\pi]) = \Gamma_\gamma$.\\
Sind $B$ und $\gamma$ wie oben, so heißt $B$ \begriff{zulässig}.
\index{zulässig}
\begin{beispiel}
 Sei $R$ konstant, also $R(t) = R > 0$, so ist $B = \overline{U_R(x_0,y_0)}$
\end{beispiel}

\begin{satz}[Integralsatz von Gauß im $\MdR^2$]
\label{Satz 13.1}
$B$ und $\gamma$ seien wie oben ($B$ also zulässig). Weiter sei $D\subseteq \MdR^2$ offen, $B\subseteq D$ und $f = (u,v) \in C^1(D,\MdR^2)$. Dann
\begin{liste}
\item $\int_B u_x(x,y)d(x,y) = \int_{\gamma} u(x,y) d(y)$
\item $\int_B v_y(x,y)d(x,y) = -\int_{\gamma} v(x,y) d(x)$
\item $\int_B \divv f(x,y)d(x,y) = \int_{\gamma} (udy - vdx)$
\end{liste}
\end{satz}

\begin{folgerung}
Mit $f(x,y) := (x,y)$ erhält man aus \ref{Satz 13.1}: Sind $B$ und $\gamma$ wie in \ref{Satz 13.1}, so gilt:
\begin{liste}
\item $\lambda_2(B) = \int_\gamma xdy$
\item $\lambda_2(B) = -\int_\gamma ydx$
\item $\lambda_2(B) = \frac12\int_\gamma (xdy - ydx)$
\end{liste}
\end{folgerung}

\begin{beispiel}
Definiere
\[B:= \{(x,y)\in\MdR^2:x^2+y^2 \le R^2\}\quad (R>0)\]
und $\gamma(t) = (R\cos t,R\sin t)$, für $t\in[0,2\pi]$, dann gilt:
\[\lambda_2(B) = \int_0^{2\pi} R\cos t\cdot R\cos t \text{ d}t = R^2\int_0^{2\pi} \cos^2t \text{ d}t = \pi R^2\]
\end{beispiel}

\begin{beweis}
Wir beweisen nur (1). ((2) beweist man analog und (3) folgt aus (1) und (2))\\
O.B.d.A: $(x_0,y_0) = (0,0)$ und $R$ stetig db. Also $\gamma = (\gamma_1,\gamma_2)$, $\gamma (t) = (\underbrace{R(t)\cos t}_{= \gamma_1(t)},\underbrace{R(t)\sin t)}_{=\gamma_2(t)}$. $R$ stetig differenzierbar. $A:= \int_B u_x(x,y)d(x,y)$\\
Zu zeigen: $A=\int_0^{2\pi} u(\gamma (t))\cdot \gamma_2'(t) dt$.\\
Mit Polarkoordinaten, Transformations-Satz und Fubini:
\begin{displaymath}
	A = \int_0^{2\pi }(\int_0^{R(t)} u_x(r\cos t,r\sin t)r dr) dt
\end{displaymath}
\begin{enumerate}
	\item $\beta(r,t) := u(r\cos t,r\sin t)$. Nachrechnen: $r\beta_r(r,t)\cos t - \beta_t(r,t)\sin t = u_x(r\cos t,r\sin t)r$. Also: 
		\begin{displaymath}
			A = \int_0^{2\pi} (\int_0^{R(t)} (r\beta_r(r,t)\cos t - \beta_t(r,t)\sin t) dr)dt
		\end{displaymath}
	\item $\int_0^{R(t)} r\beta_r(r,t) dr = r\beta(r,t)\vert_{r=0}^{r=R(t)} - \underbrace{\int_0^{R(t)} \beta(r,t) dr}_{=:\alpha(t)} = R(t)\beta(R(t),t) - \alpha(t) = R(t)u(\gamma(t)) -\alpha(t)$
	\item $\Psi(s,t) := \int_0^s \beta(r,t)dr$. Mit dem zweiten Hauptsatz aus Analysis 1 folgt: $\Psi_s(s,t) = \beta(s,t)$ \\ 7.3 \folgt $\Psi_t(s,t) = \int_0^s \beta_t(r,t) dr$.\\
		Dann: $\alpha(t) = \Psi(R(t),t)$, also 
		\begin{displaymath}
			\alpha'(t) = \Psi_s(R(t),t)\cdot R'(t) + \Psi_t(R(t),t)\cdot 1 = R'(t)\underbrace{\beta(R(t),t)}_{=u(\gamma(t))} + \int_0^{R(t)} \beta_t(r,t) dr
		\end{displaymath}
		\folgt $\int_0^{R(t)}\beta_t(r,t)dr = \alpha'(t) - R'(t)\cdot u(\gamma(t))$.
	\item Aus (1),(2),(3) folgt: \\
		\begin{align*}
		A &=  \int_0^{2\pi} (R(t)\cdot u(\gamma(t))\cdot \cos t - \alpha(t)\cos t - \alpha'(t)\sin t + R'(t)\cdot u(\gamma(t))\sin t) dt\\ &= \int_0^{2\pi}u(\gamma(t))\gamma_2'(t)dt - \int_0^{2\pi} (\alpha(t)\sin t)' dt\\ &= \int_0^{2\pi} u(\gamma(t))\gamma_2'(t)dt - \underbrace{[\alpha(t)\sin t]_0^{2\pi}}_{=0}\\ &= \int_0^{2\pi} u(\gamma(t))\gamma_2'(t) dt
		\end{align*}
\end{enumerate}
\end{beweis}

\chapter{Flächen im $\MdR^3$}
\label{Kapitel 14}


Definitionen:

\begin{longtable}{ m{2.5cm} m{3cm} m{5cm} m{6cm} }
  \textbf{Name} & Symbol & Definition & in Worten/Kommentar \\ 
  \hline 
  	& $ $ 
  	& $ $ 
  	& \\
\end{longtable}

\begin{definition}
\index{Fläche}
\index{Flächenstück}
\index{Parameterbereich}
\index{Normalenvektor}
\index{Flächeninhalt}
	Es sei $\varnothing \ne B\subseteq \MdR^2$ kompakt, $D\subseteq\MdR^2$ offen und $B\subseteq D$. Weiter sei $\varphi = (\varphi_1,\varphi_2,\varphi_3) \in C^1(D,\MdR^3)$ und $\varphi = \varphi(u,v)$. Dann heißt $\varphi_{|B}$ eine \textbf{Fläche} (im $\MdR^3$), $S:= \varphi(B)$ heißt \textbf{Flächenstück} und $B$ heißt \textbf{Parameterbereich} der Fläche. Es ist 
	\begin{displaymath}
		\varphi' = \begin{pmatrix}\frac{\partial \varphi_1}{\partial u} & \frac{\partial\varphi_1}{\partial v}\\
			\frac{\partial \varphi_2}{\partial u} & \frac{\partial\varphi_2}{\partial v}\\
			\frac{\partial \varphi_3}{\partial u} & \frac{\partial\varphi_3}{\partial v}\\
		\end{pmatrix}
	\end{displaymath}
	Sei $(u_0,v_0)\in B$ und
	\begin{align*}
	\gamma(t) &:= \varphi(t,v_0) &\gamma'(t) &= \varphi_u(t,v_0) &\gamma'(u_0) &= \varphi_u(u_0,v_0)\\
	\tilde{\gamma}(t)&:= \varphi(u_0,t) &\tilde{\gamma}'(t) &= \varphi_v(u_0,v) &\tilde{\gamma}'(v_0) &= \varphi_v(u_0,v_0)
	\end{align*}
	Definere damit den \textbf{Normalenvektor} in $\varphi(u_0,v_0)$: 
	\[N(u_0,v_0) := \varphi_u(u_0,v_0)\times\varphi_v(u_0,v_0)\]
	Seien $\Delta u,\Delta v >0$ (aber "`klein"'). $a:= \Delta u\varphi_u(u_0,v_0)$, $b:= \Delta v\varphi_v(u_0,v_0)$.
	\[P:= \{\lambda a+\mu b: \ \lambda,\mu\in [0,1]\}\] 
	Aus der Linearen Algebra folgt, der "`Inhalt"' von $P$ ist $\|a \times b\| = \Delta u\Delta v \|N(u_0,v_0)\|$.
	\begin{displaymath}
		I(\varphi) = \int_B \|N(u,v)\| d(u,v)
	\end{displaymath}
	heißt deshalb \textbf{Flächeninhalt} von $\varphi$
\end{definition}

\begin{beispiel}
	$B:=[0,2\pi]\times[-\frac\pi2,\frac\pi2]$, $D=\MdR^2$\\
	$\varphi(u,v) := (\cos u\cos v,\sin u\cos v,\sin v)$. Dann: $\varphi(B) = \{(x,y,z)\in\MdR^3:\ x^2+y^2+z^2 = 1\}$.\\
	Nachrechnen: $N(u,v) = \cos v\varphi(u,v)$. Dann: $\|N(u,v)\| = |\cos v|\underbrace{\|\varphi(u,v)\|}_{=1} = \cos v\ \ \ \ ((u,v)\in B)$. \\
	Damit gilt: 
	\[I(\varphi) = \int_B \cos v d(u,v) = \int_0^{2\pi} (\int_{-\frac\pi2}^{\frac\pi2}\cos v d(v)) d(u) = 4\pi\]
\end{beispiel}

\section{Explizite Parameterdarstellung}
Seien \(B\) und \(D\) wie in obiger Definition und \(f\in C^{1}(D,\,\mdr)\). Setze 
\[\varphi(u,v):=(u,v,f(u,v))\quad((u,v)\in D)\]
Damit ist \(\varphi_{|B}\) eine Fl\"ache (in expliziter Darstellung).
% hier Graphik einfuegen
Dann ist \(S=\varphi(B)\) gleich dem Graph von \(f_{|B}\).

\[
\varphi_{u}=(1,0,f_{u}),\quad \varphi_{v}=(0,1,f_{v}),\quad N(u,v)=(-f_{u},-f_{v},1)\quad\text{(Nachrechnen!)}
\]
Damit gilt: 
\[I(\varphi)=\int_{B}{(f_{u}^{2}+f_{v}^{2}+1)^{\frac{1}{2}}\mathrm{d}(u,v)}\]

\begin{beispiel}
Sei \(D=\mdr^{2},\,B:=\{(u,v)\in\mdr^{2}\mid u^{2}+v^{2}\leq 1\}\) und
\[f(u,v):=u^{2}+v^{2}\]
Dann ist \(\varphi(u,v)=(u,v,u^{2}+v^{2})\), \(f_{u}=2u\) und \(f_{v}=2v\). Also ist \(S=\varphi(B)\) ein Paraboloid.
\[I(\varphi)=\int_{B}{(4u^{2}+4v^{2}+1)^{\frac{1}{2}}\mathrm{d}(u,v)}\overset{\text{PK}}{=}\frac{\pi}{6}\left(\sqrt{5}^{3}-1\right)\quad \text{(Nachrechnen!)}\]
\end{beispiel}

\chapter{Integralsatz von Stokes}
\label{Kapitel 15}


Definitionen:

\begin{longtable}{ m{2.5cm} m{3cm} m{5cm} m{6cm} }
  \textbf{Name} & Symbol & Definition & in Worten/Kommentar \\ 
  \hline 
  	& $ $ 
  	& $ $ 
  	& \\
\end{longtable}

In diesem Paragraphen sei \(\varnothing\neq B\subseteq\mdr^{2}\), \(B\) kompakt, \(D\subseteq\mdr^{2}\) offen, \(B\subseteq D\)
und \(\varphi=(\varphi_{1},\varphi_{2},\varphi_{3})\in C^{1}(D,\mdr^{3})\). Das hei\ss t: \(\varphi_{|B}\) ist eine Fl\"ache mit 
Parameterbereich \(B\), \(S:=\varphi(B)\)

\begin{definition}
\index{Oberflächenintegral}
Definiere die folgenden \textbf{Oberfl\"achenintegrale}:
\begin{enumerate}
\item Sei \(f:\,S\to\mdr\) stetig. Dann: 
\[
\int_{\varphi}{f\mathrm{d}\sigma}:=\int_{B}{f(\varphi(u,v))\lVert N(u,v)\rVert\mathrm{d}(u,v)}
\]
\item Sei \(F:\,S\to\mdr^{3}\) stetig. Dann:
\[
\int_{\varphi}{F\cdot n\mathrm{d}\sigma}:=\int_{B}{F(\varphi(u,v))\cdot N(u,v)\mathrm{d}(u,v)}
\]
\end{enumerate}
\end{definition}

\begin{beispiel}
Seien \(D,\,B,\,f,\,\varphi\) wie im letzten Beispiel in Kapitel 14.	% Paragraphenzeichen!?

Sei \(F(x,y,z):=(x,y,z)\); bekannt: \(N(u,v)=(-2u,-2v,1)\). Dann:
\begin{align*}
F(\varphi(u,v))\cdot N(u,v)&=F(u,v,u^{2}+v^{2})\cdot(-2u,-2v,1)\\
&=(u,v,u^{2}+v^{2})\cdot (-2u,-2v,1)\\
&=-(u^{2}+v^{2})
\end{align*}

Also: 
\[
\int_{\varphi}{F\cdot n\mathrm{d}\sigma}=-\int_{B}{(u^{2}+v^{2})\mathrm{d}(u,v)}=-\frac{\pi}{2}
\]
\end{beispiel}

\begin{satz}[Integralsatz von Stokes]
\label{Satz 15.1}
Es sei \(B\) zul\"assig, \(\partial B=\Gamma_{\gamma}\), wobei \(\gamma=(\gamma_{1},\gamma_{2})\) wie zu Beginn des Paragraphen
13 ist. Es sei \(\varphi\in C^{2}(D,\mdr^{3})\). Weiter sei \(G\subseteq\mdr^{3}\) offen, \(S\subseteq G\) und \(F=(F_{1},F_{2},F_{3})\in C^{1}(G,\mdr^{3})\). Dann:
\[
\underbrace{\int_{\varphi}{\rot F\cdot n\mathrm{d}\sigma}}_{\text{Oberfl\"achenint.}}=
    \underbrace{\int_{\varphi\circ\gamma}{F(x,y,z)\cdot\mathrm{d}(x,y,z)}}_{\text{Wegint.}}
\]
\end{satz}

\begin{beispiel}
\(D,\,B,\,f,\,F\) und \(\varphi\) seien wie in obigem Beispiel.
% Bild einfuegen...
Hier: \(\gamma(t)=(\cos t,\sin t)\quad(t\in [0,2\pi])\). 
Dann: \((\varphi\circ\gamma)(t)=\varphi(\cos t, \sin t)=(\cos t, \sin t, 1)\quad(t\in [0,2\pi])\).

Es ist \(\rot F=0\), also: \(\int_{\varphi}{\rot F\cdot n\mathrm{d}\sigma}=0\)
\begin{align*}
\int_{\varphi\circ\gamma}{F(x,y,z)\mathrm{d}(x,y,z)}&=
    \int_{0}^{2\pi}{F((\varphi\circ\gamma)(t))\cdot(\varphi\circ\gamma)'(t)\mathrm{d}t}\\
&=\int_{0}^{2\pi}{F(\cos t,\sin t, 1)\cdot (-\sin t,\cos t,0)\mathrm{d}t}\\
&=\int_{0}^{2\pi}{\underbrace{(\cos t,\sin t,1)\cdot (-\sin t,\cos t,0)}_{=0}\mathrm{d}t}\\
&=0
\end{align*}
\end{beispiel}

\begin{beweis}
Sei \(\varphi:=\varphi\circ\gamma,\,\varphi=(\varphi_{1},\varphi_{2},\varphi_{3})\), also 
    \(\varphi_{j}=\varphi_{j}\circ\gamma\quad(j=1,2,3)\).

Zu zeigen:
\begin{align*}
\int_{\varphi}{\rot F\cdot n\mathrm{d}\sigma}
    &=\int_{\varphi}{F(x,y,z)\mathrm{d}(x,y,z)}\\
    &=\int_{0}^{2\pi}{F(\varphi(t))\cdot\varphi'(t)\mathrm{d}t}\\
    &=\int_{0}^{2\pi}{\left(\sum_{j=1}^{3}{F_{j}(\varphi(t))\varphi_{j}'(t)}\right)\mathrm{d}t}\\
    &=\sum_{j=1}^{3}{\int_{0}^{2\pi}{F_{j}(\varphi(t))\varphi_{j}'(t)\mathrm{d}t}}
\end{align*}

Es ist \(\int_{\varphi}{\rot F\cdot n\mathrm{d}\sigma}=\int_{B}{\underbrace{(\rot F)(\varphi(x,y))\cdot(\varphi_{x}(x,y)\times\varphi_{y}(x,y))}_{=:g(x,y)}\mathrm{d}(x,y)}\).
F\"ur \(j=1,2,3\):
\[
h_{j}(x,y):=\left(\underbrace{F_{j}(\varphi(x,y))\frac{\partial\varphi_{j}}{\partial y}(x,y)}_{=:u_{j}(x,y)},\underbrace{-F_{j}(\varphi(x,y))\frac{\partial\varphi_{j}}{\partial x}(x,y)}_{=:v_{j}(x,y)}\right)\quad((x,y)\in D)
\]


\(h_{j}=(u_{j},v_{j});\quad F\in C^{1},\,\varphi\in C^{2}\), damit folgt: \(h_{j}\in C^{1}\)

Nachrechnen: \(g=\mathrm{div} h_{1}+\mathrm{div} h_{2}+\mathrm{div} h_{3}\)

Damit:
\begin{align*}
\int_{B}{\rot F\cdot n\mathrm{d}\sigma}
    &=\sum_{j=1}^{3}{\int_{B}{\mathrm{div}\,h_{j}(x,y)\mathrm{d}(x,y)}}\\
    &=\sum_{j=1}^{3}{\int_{\gamma}{(u_{j}\mathrm{d}y-v_{j}\mathrm{d}x)}}\\
    &=\int_{0}^{2\pi}{F_{j}(\varphi(t))\varphi_{j}'(t)\mathrm{d}t}
\end{align*}
\end{beweis}

\chapter{$\fl^{p}$-R\"aume und $\mathrm{L}^{p}$-R\"aume}
\label{Kapitel 16}


Definitionen:

\begin{longtable}{ m{2.5cm} m{3cm} m{5cm} m{6cm} }
  \textbf{Name} & Symbol & Definition & in Worten/Kommentar \\ 
  \hline 
  	& $ $ 
  	& $ $ 
  	& \\
\end{longtable}

Stets in diesem Paragraphen: \(\varnothing\neq X\in\fb_{d}\)

\begin{definition}
Sei \(p\in[1,+\infty]\).
\[
p':=\begin{cases}
\infty&,\,p=1\\
1&,\,p=\infty\\
\frac{p}{p-1}&,\,1<p<\infty
\end{cases}
\]
Dann gilt: \(\frac{1}{p}+\frac{1}{p'}=1\) und \(p=p'\Leftrightarrow p=2\).
\end{definition}

\begin{hilfssatz}
Seien \(x,y\geq 0,\,p\in(1,\infty)\), dann gilt: \(xy\leq\frac{x^{p}}{p}+\frac{y^{p'}}{p'}\)
\end{hilfssatz}
\begin{beweis}
F\"ur \(t>0:\,f(t):=\frac{t}{p}+\frac{1}{p'}-t^{\frac{1}{p}}\)

\"Ubung: \(\min\{f(t)\mid t>0\}=f(1)=0\)

D.h.: \(t^{\frac{1}{p}}\leq\frac{t}{p}+\frac{1}{p'}\quad\forall t>0\)

Seien \(u,v>0,\,t:=\frac{u}{v}\). Dann: \(\frac{u^{\frac{1}{p}}}{v^{\frac{1}{p}}}\leq\frac{u}{vp}+\frac{1}{p'}\). Daraus folgt
\(u^{\frac{1}{p}}v^{1-\frac{1}{p}}\leq\frac{u}{p}+\frac{v}{p'}\implies u^{\frac{1}{p}}v^{\frac{1}{p'}}\leq \frac{u}{p}+\frac{v}{p'}\)

Seien \(x,y>0:\,u:=x^{p},\,v:=y^{p'}\). Dann: \(xy\leq\frac{x^{p}}{p}+\frac{y^{p'}}{p'}\).

Im Falle \(x=0\) oder \(y=\infty\) ist die Ungleichung trivialerweise richtig.
\end{beweis}

\begin{erinnerung}
Sei \(f:\,X\to\mdr\) messbar und \(p>0\), so ist \(\lvert f\rvert^{p}\) messbar (vgl. Kapitel 3).

Es gilt: \(\lvert f\rvert^{p}\in\fl^{1}(X)\Leftrightarrow \int_{X}{\lvert f\rvert^{p}\mathrm{d}x}<\infty\)
\end{erinnerung}

\begin{definition}
\begin{enumerate}
\item Sei \(p\in[1,\infty)\). \(\fl^{p}(X)=\{f:\,X\to\mdr\mid f \text{ ist messbar und }\int_{X}{\lvert f\rvert^{p}\mathrm{d}x<\infty}\}\).

F\"ur \(f\in\fl^{p}(X)\): \(\lVert f\rVert_{p}=\left(\int_{X}{\lvert f\rvert^{p}\mathrm{d}x}\right)^{\frac{1}{p}}\)
\item \(\fl^{\infty}(X)=\{f:\,X\to\mdr\mid f\text{ ist messbar und }f\text{ ist f.\"u. beschr\"ankt}\}\)

F\"ur \(f\in\fl^{\infty}(X)\): \(\lVert f\rVert_{\infty}:=\esssup_{x\in X}\lVert f(x)\rVert=\inf\{c>0\mid \exists\text{Nullmenge }N_{c}\subseteq X: \lvert f(x)\rvert\leq c\,\forall x\in X\setminus N_{c}\}\)
\end{enumerate}
\end{definition}

\begin{bemerkung}
Es sei \(f\in\fl^{\infty}(X)\) und stetig. Außerdem habe jede in \(X\) offene, nichtleere Teilmenge positives Ma\ss. Dann ist \(f\) auf \(X\) beschr\"ankt und \(\sup_{x\in X}\lvert f(x)\rvert=\esssup_{x\in X}\lvert f(x)\rvert\). 
\end{bemerkung}
\begin{beweis}
\"Ubung (ist \(N\subseteq X\) eine Nullmenge, so ist \(N^{\circ}=\varnothing\) und \(\overline{X\setminus N}=X\))
\end{beweis}

\begin{beispiel}
Sei \(d=1,\,X=[1,\infty),\,p>1\,(p<\infty),\,\alpha,\beta>0,\,f(x)=\frac{1}{x^{\alpha}},\,g(x)=\frac{1}{x^{\beta}}\)
\begin{enumerate}
\item \[f\in\fl^{p}(X)\overset{\text{\ref{Satz 4.14}}}{\iff}\int_{1}^{\infty}{\frac{1}{x^{\alpha p}}}\mathrm{d}x\]	
konvergiert genau dann, wenn \(\alpha p>1\Leftrightarrow \alpha>\frac{1}{p}\)
\item
\[fg\in\fl^{1}(X)\overset{\text{\ref{Satz 4.14}}}{\iff}\int_{1}^{\infty}{\frac{1}{x^{\alpha+\beta}}\mathrm{d}x}\]	
konvergiert genau dann, wenn $\alpha+\beta >1$
\end{enumerate}
\end{beispiel}

\begin{satz}
\label{Satz 16.1}
Sei \(p\in[1,\infty]\) und \(p'\) wie zu Anfang dieses Kapitels, also \(\frac{1}{p}+\frac{1}{p'}=1\).
\begin{enumerate}
\item Sei \(f\in\fl^{p}(X)\) und \(g\in\fl^{p'}(X)\).
\index{Ungleichung!Hölder}
Dann ist \(fg\in\fl^{1}(X)\) und es gilt die \textbf{Höldersche Ungleichung}:
\[
\lVert fg\rVert_{1}\leq\lVert f\rVert_{p}\cdot\lVert g\rVert_{p'}
\]

\index{Ungleichung!Cauchy-Schwarz}
Ist \(p=2\,(\implies p'=2)\), so hei\ss t obige Ungleichung auch \textbf{Cauchy-Schwarzsche Ungleichung}.
\item \(\fl^{p}(X)\) ist ein reeller Vektorraum und f\"ur \(f,g\in\fl^{p}(X)\) gilt die \textbf{Minkowskische Ungleichung}:
\index{Ungleichung!Minkowski}
\[
\lVert f+g\rVert_{p}\leq\lVert f\rVert_{p}+\lVert g\rVert_{p}
\]
\end{enumerate}
\end{satz}

\begin{beweis}
\begin{enumerate}
\item Unterscheide die folgenden F\"alle:
\begin{itemize}
\item[Fall 1:]  \(p=1\) (also \(p'=\infty\)) oder \(p=\infty\) (also \(p'=1\)). Etwa \(p=1,\,p'=\infty\).

Sei \(c>0\) und \(N_{c}\subseteq X\) Nullmenge mit: \(\lvert g(x)\rvert\leq c\,\forall x\in X\setminus N_{c}\). 
\(\tilde{g}:=\mathds{1}_{X\setminus N_{c}}\cdot g\)

Dann: \(g=\tilde{g}\) fast \"uberall und \(\lvert\tilde{g}\rvert\leq c\) auf \(X\). Weiter: \(fg=f\tilde{g}\) fast \"uberall,
bzw. \(\lvert fg\rvert=\lvert f\tilde{g}\rvert\) fast \"uberall.

Dann:
\[
\int_{X}{\lvert fg\rvert\mathrm{d}x}=\int_{X}{\lvert f\tilde{g}\rvert\mathrm{d}x}=\int_{X}{\lvert f\rvert\underbrace{\lvert\tilde{g}\rvert}_{\leq c}\mathrm{d}x}\leq\int_{X}{\lvert f\rvert\mathrm{d}x}=c\cdot\lVert f\rVert_{1}<\infty
\]
Also: \(fg\in\fl^{1}(X)\) und \(\lVert fg\rVert_{1}\leq c\lVert f\rVert_{1}\). \"Ubergang zum Infimum \"uber alle \(c>0\) 
liefert: \(\lVert fg\rVert_{1}\leq\lVert g\rVert_{\infty}\cdot\lVert f\rVert_{1}\)
\item[Fall 2:] Sei \(1<p<\infty\). Ist \(\lVert f\rVert_{p}=0\) oder \(\lVert g\rVert_{p'}=0\), so ist \(f=0\) fast \"uberall
oder \(g=0\) fast \"uberall. Daraus folgt: \(\lvert fg\rvert=0\) fast \"uberall.
Mit \ref{Satz 5.2} folgt: \(\int_{X}{\lvert fg\rvert\mathrm{d}x}=0\). Daraus folgen die Behauptungen.


Sei \(\lVert f\rVert_{p}>0\) und \(\lVert g\rVert_{p'}>0\).

Aus obigem Hilfssatz:
\[
\frac{\lvert f(x)\rvert}{\lVert f\rVert_{p}}\cdot\frac{\lvert g(x)\rvert}{\lVert g\rVert_{p'}}\leq\frac{1}{p}\frac{\lvert f(x)\rvert^{p}}{\lVert f\rVert_{p}^{p}}+\frac{1}{p'}\frac{\lvert g(x)\rvert^{p'}}{\lVert g\rVert_{p'}^{p'}}\quad\forall x\in X
\]
Integration liefert:
\begin{align*}
\frac{1}{\lVert f\rVert_{p}\cdot\lVert g\rVert_{p'}}\int_{X}{\lvert f(x)g(x)\rvert\mathrm{d}x}
	&\leq\frac{1}{p}\cdot\frac{1}{\lVert f\rVert_{p}^{p}}\int_{X}{\lvert f\rvert^{p}\mathrm{d}x}+
	\frac{1}{p'}\cdot\frac{1}{\lVert g\rVert_{p'}^{p'}}\int_{X}{\lvert g\rvert^{p'}\mathrm{d}x}\\
	&=\frac{1}{p}+\frac{1}{p'}\\
	&=1<\infty
\end{align*}
Daraus folgt: \(fg\in\fl^{1}(X)\) und
\[
\frac{\lVert fg\rVert_{1}}{\lVert f\rVert_{p}\cdot\lVert g\rVert_{p}}\leq 1\Leftrightarrow \lVert fg\rVert_{1}\leq\lVert f\rVert_{p}\cdot\lVert g\rVert_{p}
\]
\end{itemize}
\item Klar: Ist \(f\in\fl^{p}(X)\) und \(\alpha\in\mdr\), so ist \(\alpha f\in\fl^{p}(X)\)
\begin{itemize}
\item[Fall 1:] \(p=1\): Mit \ref{Satz 4.11} folgt: \(\fl^{1}(X)\) ist ein reeller Vektorraum.

Seien \(f,g\in\fl^{1}(X)\). Dann: \(\lvert f+g\rvert\leq\lvert f\rvert+\lvert g\rvert\) auf \(X\). Damit:
\[
\int_{X}{\lvert f+g\rvert\mathrm{d}x}\leq\int_{X}{\lvert f\rvert\mathrm{d}x}+\int_{X}{\lvert g\rvert\mathrm{d}x}
\]
\item[Fall 2:] \(p=\infty\): Seien \(f,\,g\in\fl^{\infty}(X)\). Seien \(c_{1},\,c_{2}>0\) und \(N_{1},\,N_{2}\subseteq X\)
Nullmengen und \(\lvert f(x)\rvert\leq c_{1}\forall x\in X\setminus N_{1},\,\lvert g(x)\rvert\leq c_{2}\forall x\in X\setminus N_{2}\).

\(N=N_{1}\cup N_{2}\) ist eine Nullmenge. Dann: \(\lvert f(x)+g(x)\rvert\leq\lvert f(x)\rvert+\lvert g(x)\rvert\leq c_{1}+c_{2}
\forall x\in X\setminus N\). Es folgt: \(f+g\in\fl^{\infty}(X)\) und \(\lVert f+g\rVert_{\infty}\leq c_{1}+c_{2}\).

\"Ubergang zum Infimum \"uber alle solche \(c_{1}\), bzw. \(c_{2}\), liefert: \(\lVert f+g\rVert_{\infty}\leq\lVert f\rVert_{\infty}+\lVert g\rVert_{\infty}\).
\item[Fall 3:] Sei \(1<p<\infty\) und \(f,\,g\in\fl^{p}(X)\). Es ist \(\lvert f+g\rvert^{p}\leq(\lvert f\rvert+\lvert g\rvert)^{p}\leq\left(2\max\{\lvert f\rvert,\,\lvert g\rvert\}\right)^{p}\leq 2^{p}\left(\lvert f\rvert^{p}+\lvert g\rvert^{p}\right)\)
auf \(X\). Mit \ref{Satz 4.9} folgt: \(\lvert f+g\rvert^{p}\in\fl^{1}(X)\implies f+g\in\fl^{p}(X)\)\\

\(p'=\frac{p}{p-1};\,h:=\lvert f+g\rvert^{p-1}\), dann: \(h^{p'}=\left(\lvert f+g\rvert^{p-1}\right)^{\frac{p}{p-1}}=\lvert f+g\rvert^{p}\in\fl^{1}(X)\). Dann ist \(h\in\fl^{p'}(X)\). Also: \(h\in\fl^{p'}(X),\,f\in\fl^{p}(X)\) 
(und \(\frac{1}{p}+\frac{1}{p'}=1\)).

Mit der H\"olderschen Ungleichung folgt:
\(\lVert f\cdot f_{1}\rVert\leq\lVert f\rVert_{p}\lVert h\rVert_{p'}\implies\int_{X}{h\lvert f\rvert\mathrm{d}x}\leq\lVert f\rVert_{p}\left(\int_{X}{h^{p'}\mathrm{d}x}\right)^{\frac{1}{p'}}\). Dann:
\begin{align*}
\int_{X}{\lvert f\rvert\lvert f+g\rvert^{p-1}\mathrm{d}x}
    &\leq\lVert f\rVert_{p}\left(\int_{X}{\left(\lvert f+g\rvert^{p-1}\right)^{p'}\mathrm{d}x}\right)^{\frac{1}{p'}}\\
    &=\lVert f\rVert_{p}\left(\lVert f+g\rVert_{p}^{p}\right)^{\frac{1}{p'}}\\
    &=\lVert f\rVert_{p}\lVert f+g\rVert_{p}^{p-1}
\end{align*}

Genauso: \(\int_{X}{\lvert g\rvert\lvert f+g\rvert^{p-1}\mathrm{d}x}\leq\lVert g\rVert_{p}\lVert f+g\rVert_{p}^{p+1}\)

Dann:
\begin{align*}
\lVert f+g\rVert_{p}^{p}&=\int_{X}{\lvert f+g\rvert^{p}\mathrm{d}x}\\
    &=\int_{X}{\lvert f+g\rvert\lvert f+g\rvert^{p-1}\mathrm{d}x}\\
    &=\int_{X}{\lvert f\rvert\lvert f+g\rvert^{p-1}\mathrm{d}x}+\int_{X}{\lvert g\rvert\lvert f+g\rvert^{p-1}\mathrm{d}x}\\
    &\leq\left(\lVert f\rVert_{p}+\lVert g\rVert_{p}\right)\lVert f+g\rVert_{p}^{p-1}
\end{align*}

Teilen durch \(\lVert f+g\rVert_{p}^{p-1}\) liefert die Minkowski-Ungleichung.

\end{itemize}
\end{enumerate}
\end{beweis}

\begin{satz}
\label{Satz 16.2}
Sei $\lambda_d(X)<\infty$, $p,q\ge 1$ und $p\leq q \leq \infty$. Dann ist $\fl^q(X)\subseteq\fl^p(X)$ und es gilt:
\[\forall f\in\fl^q(X): \|f\|_p\le\lambda_d(X)^{\frac1p-\frac1q}\|f\|_q\]
\end{satz}

\begin{beweis}
Sei $f\in\fl^q(X)$.\\
\textbf{Fall $p=q$:} Klar.\\
\textbf{Fall $q=\infty$:} Leichte Übung!\\
\textbf{Fall $p<q<\infty$:}\\
Sei $r:=\frac qp>1$, dann ist $\frac 1{r'}=1-\frac pq$. Aus $|f|^{pr}=|f|^q\in\fl^1(X)$ folgt $|f|^p\in\fl^r(X)$. Definiere $g:=\mathds{1}_X$, dann ist $g\in\fl^{r'}(X)$, da $\lambda_d(X)<\infty$. Wegen \ref{Satz 16.1} gilt dann:
\[g\cdot|f|^p\in\fl^1(X)\implies |f|^p\in\fl^1(X)\implies f\in\fl^p(X)\]
Aus der Hölderschen Ungleichung folgt:
\begin{align*}
\|f\|^p_p&=\|g\cdot |f|^p\|_1\\
&\le \|g\|_{r'}\cdot\||f|^p\|_r\\
&= (\int_X g^{r'}\text{ d}x)^{\frac 1{r'}}\cdot(\int_X |f|^{pr}\text{ d}x)^{\frac 1r}\\
&= \lambda_d(X)^{\frac1{r'}}\cdot(\int_X |f|^{q}\text{ d}x)^{\frac pq}\\
&= \lambda_d(X)^{1-\frac pq}\cdot\|f\|^p_q
\end{align*}
Also gilt:
\[\|f\|_p\le\lambda_d(X)^{\frac1p-\frac1q}\|f\|_q\]
\end{beweis}

\begin {beispiel}
\begin{enumerate}
\item Sei $X:=(0,1]$, $1\le p<q<\infty$ (also $\frac 1q<\frac1p$) und $f(x):=\frac 1{x^\alpha}$ $(\alpha>0)$. Dann gilt nach 
\ref{Satz 4.14} und Analysis I:
\begin{align*}
f\in\fl^p(X)&\iff\int_0^1\frac1{x^{\alpha p}}\text{ d}x \text{ konvergiert}\\
&\iff\alpha p<1\\
&\iff \alpha<\frac 1p
\end{align*}
Sei $\frac 1q<\alpha<\frac 1p$, dann ist $f\in\fl^p(X)$ und $f\not\in\fl^q(X)$. D.h. $\fl^p(X)\not\subseteq\fl^q(X)$ und aus \ref{Satz 16.2} folgt $\fl^q(X)\subseteq\fl^p(X)$.
\item Sei $X:=[1,\infty)$, $p=1$, $q\in(1,\infty)$ und $f(x):=\frac 1x$. Dann gilt nach \ref{Satz 4.14} und Analysis I: $f\not\in\fl^p(X)$ und $f\in\fl^q(X)$. D.h. also $\fl^q(X)\not\subseteq\fl^p(X)$.\\
Definiere $g(x):=\mathds{1}_{[1,2)}\cdot (2-x)^{-\frac 1q}$. Übung: $g\in\fl^p(X)$ und $g\not\in\fl^q(X)$. D.h. also $\fl^p(X)\not\subseteq\fl^q(X)$.
\end{enumerate}
\end{beispiel}

\begin{satz}[Satz von Lebesgue ($\fl^p$-Version)]
\label{Satz 16.3}
Sei $1\le p<\infty$, $f:X\to\mdr$ sei messbar, $g:X\to[0,\infty]$ integrierbar und $(f_n)$ eine Folge in $\fl^p(X)$ mit den Eigenschaften:
\begin{enumerate}
\item $f_n\to f$ f.ü. auf $X$
\item $\forall n\in\mdn: |f_n|^p\le g$ f.ü. auf $X$.
\end{enumerate}
Dann ist $f\in\fl^p(X)$ und es gilt
\[\|f_n-f\|_p\stackrel{n\to\infty}\to 0\]
\end{satz}

\begin{beweis}
Aus (i) und (ii) folgt: $|f|^p \leq g$ f.\"u.
Im Paragraphen 5 haben wir gesehen, dass dann gilt:
\[ \int_X |f|^p \text{ d}x \leq \int_X g \text{ d}x < \infty \]
(denn $g$ ist nach Voraussetzung integrierbar). 
Daraus folgt: $f \in \fl^p(X)$.

Setze $g_n := |f_n - f|^p$. Aus (i): $g_n \to 0$ f.\"u. Es sind $f_n, f \in \fl^p(X)$ (ersteres nach Voraussetzung, zweiteres haben wir gerade gezeigt), und weil $\fl^p(X)$ ein reeller Vektorraum ist (\ref{Satz 16.1}(2)), folgt:
\[ f_n - f \in \fl^p(X) \]
Also $g_n \in \fl^1(X)$.
Es ist 
\[ 0 \leq g_n \leq \left( |f_n| + |f| \right)^p \leq \left( g^{\frac{1}{p}} + g^{\frac{1}{p}} \right)^p = \left( 2g^{\frac{1}{p}} \right)^p = 2^p g \quad\text{f.\"u.} \]
Mit \ref{Satz 6.2} folgt schließlich:
\[ \underbrace{\int_X g_n \text{ d}x}_{=\|f_n - f\|_p^p} \to 0. \]
\end{beweis}

Aus \ref{Satz 16.1} folgt: $\fl^p(X)$ ist ein reeller Vektorraum (VR), wobei für $f,g\in\fl^p(X)$ gilt:
\[\|\alpha f\|_p=|\alpha|\cdot \|f\|_p\quad (\alpha\in\mdr)\]
\[\|f+g\|_p\le\|f\|_p+\|g\|_p\]
Aber $\|\cdot\|_p$ ist \textbf{keine} Norm auf $\fl^p(X)$! Denn aus $\|f\|_p=0$ folgt nur $f=0$ f.ü.

\begin{definition}
Es sei $\cn:=\{f:X\to\mdr\mid f\text{ ist messbar und } f=0 \text{ f.ü.}\}$, dann ist $\cn$ ein Untervektorraum von $\fl^p(X)$. Definiere
\[L^p(X):=\fl^p(X)\diagup\cn=\{\hat f=f+\cn\mid f\in\fl^p(X)\}\]
Aus der Linearen Algebra ist bekannt, dass $L^p(X)$ durch die Skalarmultiplikation
\[\alpha\cdot\hat f := \widehat{\alpha f}\]
und die Addition
\[\hat f+\hat g:=\widehat{f+g}\]
zu einem Vektorraum über $\mdr$ wird.
\end{definition}

Setze f\"ur $\hat f \in L^1(X)$: 
\[\int_X \hat f(x) \text{ d}x := \int_X f(x) \text{ d}x\]
dabei ist diese Definition unabh\"angig von der Wahl des Repr\"asentanten $f \in \fl^1(X)$ von $\hat f$, denn: ist auch noch $g \in \fl^1(X)$ und $\hat g = \hat f$, so ist $f - g \in \cn$, also $f-g = 0$ f.\"u. und damit: $\int_X f \text{ d}x = \int_X g \text{ d}x$.

F\"ur $\hat f \in L^p(X)$ definiere 
\[\| \hat f \|_p := \| f \|_p\]
wobei diese Definition unabh\"angig ist von der Wahl des Repr\"asentanten $f \in \fl^p(X)$ von $\hat f$.

F\"ur $\hat f, \hat g \in L^2(X)$ setze 
\[( \hat f | \hat g ) := \int_X f(x)g(x) \text{ d}x\]
(auch diese Definition ist Repr\"asentanten-unabh\"angig) (Beachte: $f\cdot g \in \fl^1(X)$ )

\textbf{Dann gilt:} 
\index{Ungleichung!Cauchy-Schwarz}
\begin{enumerate} \item $L^p(X)$ ist unter $\| \cdot \|_p$ ein normierter Raum (NR).
\item F\"ur $\hat f, \hat g \in L^2(X)$ gilt:
\[ | ( \hat f | \hat g ) | = | \int_X f(x)g(x) \text{ d}x | \leq \int_X |fg| \text{ d}x = \| fg \|_1 \overset{\ref{Satz 16.1}}{\leq} \| f \|_2 \| g \|_2 = \| \hat f \|_2 \| \hat g \|_2 \]
\textbf{(Cauchy-Schwarzsche Ungleichung)}
\end{enumerate}
\textbf{Nachrechnen:} $( \hat f | \hat g )$ definiert ein Skalarprodukt auf $L^2(X)$. Es gilt:
\[ ( \hat f | \hat f) = \int_X f(x)^2 \text{ d}x = \| \hat f \|_2^2 \]
\textbf{Also:} $\| \hat f \|_2 = \sqrt{( \hat f | \hat f )}$

\begin{definition}
\index{Prähilbertraum}
\index{Hilbertraum}
Sei $(B, \| \cdot \|)$ ein normierter Raum. Gilt mit einem Skalarprodukt $( \cdot | \cdot )$ auf $B$:
\begin{align*}
\tag{$*$} \| v \| = \sqrt{(v | v)} \quad \forall v \in B
\end{align*}
so hei\ss t $B$ ein \textbf{Prähilbertraum}. Ist $B$ ein Banachraum mit $(*)$, so hei\ss t $B$ ein \textbf{Hilbertraum}.
\end{definition}

\textbf{Vereinbarung:} ab jetzt sei stets in diesem Paragraphen $1 \leq p < \infty$.

\begin{bemerkung}
\index{Chauchyfolge}
Seien \(f,f_n\in\fl^p(X)\)
\begin{enumerate}
\item 	\(\| f_n-f\|_p = \| \hat{f_n}-\hat f\|_p\to 0\) genau
		dann, wenn \((\hat{f_n})\) eine konvergente Folge im normierten Raum \(L^p(X)\) 
 		mit dem Grenzwert \(\hat f\) ist.
\item 	\((\hat f_n)\) ist eine \textbf{Cauchyfolge} (CF) in \(L^p(X)\) genau dann, wenn für jedes $\ep>0$ ein $n_0\in\mdn$ exitiert mit:
		\begin{align*}
		\tag{$*$} \| \hat f_n-\hat f_m\|_p	=\| f_n-f_m\|_p<\ep\quad\forall n,m\geq n_0
		\end{align*}
\item 	Wie in Analysis II zeigt man: gilt \(\| f_n-f\|_p=
		\| \hat f_n-\hat f\|_p\to 0\), so ist \((\hat f_n)\) eine Cauchyfolge
		in \(L^p(X)\).


\end{enumerate}
\end{bemerkung}

\begin{satz}[Satz von Riesz-Fischer]
\label{Satz 16.4}
\((\hat f_n)\) sei eine Cauchyfolge in \(L^p(X)\), das heißt es gilt \((\ast)\) aus obiger Bemerkung (2).
Dann existiert ein \(f\in\fl^p(X)\) und eine Teilfolge \((f_{n_j})\) von \((f_n)\) mit:
\begin{enumerate}
\item 	\(f_{n_j}\to f\) fast überall auf \(X\).
\item 	\(\| f_n-f\|_p\to 0 \ \ (n\to\infty)\).
\end{enumerate}
Das heißt \(L^p(X)\) ist ein Banachraum (\(L^2(X)\) ist ein Hilbertraum).
\end{satz}

\begin{bemerkung}
Voraussetzungen und Bezeichnungen seien wie in \ref{Satz 16.4}. Im Allgmeinen wird \textbf{nicht}
gelten, dass fast überall \(f_n\to f\) ist.
\end{bemerkung}

\begin{beispiel}
Sei \(X=[0,1]\) und \((I_n)\) sei die folgende Folge von Intervallen:
\[I_1=\left[0,1\right], I_2=\left[0,\frac12\right], I_3=\left[\frac12,1\right], I_4=\left[0,\frac14\right],
I_5=\left[\frac14,\frac12\right], I_6=\left[\frac12, \frac34\right], I_7=\left[\frac34,1\right], \ldots\]
Es sei \(f_n:=\mathds{1}_{I_n}\), sodass \(\int_X f_n\,dx=\int_{I_n}1\,dx=\lambda_1(I_n)\to 0\).
Also \(\hat f_n\in L^1(X)\) und \(\| \hat f_n-\hat 0\|_1\to 0\). 
Ist \(x\in X\), so gilt: \(x\in I_n\) für unendlich viele \natn. Daraus folgt, dass eine Teilfolge 
\(I_{n_j}\) mit \(x\in I_{n_j}\) für jedes \(j\in\mdn\) existiert. Somit ist \(f_{n_j}(x)=1\) für jedes \(j\in\mdn\)
und deshalb gilt fast überall \(f_n\nrightarrow 0\).
\end{beispiel}

\begin{beweis}[von \ref{Satz 16.4}]
Setze \(\ep_j:=\frac1{2^j}\ (j\in\mdn)\). 
Zu \(\ep_1\) existiert ein \(n_1\in\mdn\) mit \(\| f_l-f_{n_1}\|_p<\ep_1\) 
für alle \(l\geq n_1\). 
Zu \(\ep_2\) existiert ein \(n_2\in\mdn\) mit \(n_2>n_2\) und 
\(\| f_l-f_{n_2}\|_p<\ep_2\) für alle \(l\geq n_2\).
Etc.\\
Wir erhalten eine Teilfolge \((f_{n_j})\) mit 
\[(+)\ \ \ \| f_l-f_{n_j}\|_p<\ep_j \text{ für alle } l\geq n_j \text{ mit } j\in\mdn\]
Setze \(g_j:=f_{n_{j+1}}-f_{n_j}\ (j\in\mdn)\). Klar: \(g_l\in\fl^p(X)\).
Für \(N\in\mdn\): \[S_N:=\int_X\left(\sum^N_{j=1}\lvert g_j(x)\rvert^p\right)^{\frac1p}\]
Dann:
\begin{align*}
	S_N=\left\lvert\left\lvert\sum^N_{j=1}\lvert g_j\rvert\right\rvert\right\rvert_p 
	\leq \sum^N_{j=1}\| g_j\|_p
	\overset{\text{(+)}}\leq \sum^N_{j=1}\ep_j
	=\sum^N_{j=1}\frac1{2^j}
	\leq 1
\end{align*}
Setze \[g(x):=\sum^\infty_{j=1}\lvert g_j(x)\rvert \text{ für } x\in X\]
Es ist \(g\geq0\) und messbar. Weiter gilt:
\begin{align*}
	0\leq \int_X g^p\,dx
	=\int_X\lim_{N\to\infty}\left(\sum^N_{j=1}\lvert g_j\rvert\right)^p\,dx
	\overset{\ref{Satz 6.2}}\leq \liminf_{N\to\infty}S_N^p
	\leq 1
\end{align*}
Somit ist \(g^p\) ist integrierbar. Aus \ref{Satz 5.2} folgt, dass eine Nullmenge \(N_1\subseteq X\)
existiert mit \(0\leq g^p(x)<\infty\) für alle \(x\in X\setminus N_1\). Es ist dann auch
\(0\leq g(x)<\infty\) für alle \(x\in X\setminus N_1\) und somit folgt nach Konstruktion von $g$, dass
\(\sum^\infty_{j=1}g_j\,dx\) konvergiert absolut in jedem \(x\in X\setminus N_1\).
Aus Analysis I folgt, dass damit \(\sum^\infty_{j=1}g_j\,dx\) in jedem 
\(x\in X\setminus N_1\) konvergiert.

Für \(m\in\mdn\): 
\[\sum^{m-1}_{j=1}g_j=f_{n_m}-f_{n_1} \implies f_{n_m}=\sum^{m-1}_{j=1}g_j + f_{n_1} \]
Deshalb ist \((f_{n_m})\) konvergent (in \mdr) für alle \(x\in X\setminus N_1\).
\begin{align*}
f(x):=
	\begin{cases}
	\lim_{m\to\infty}f_{n_m}(x) 	&, x\in X\setminus N_1 \\
	0 						&, x\in N_1
	\end{cases}
\end{align*}
Aus \S 3 ist bekannt, dass $f$ messbar ist. Klar: \(f_{n_m}\to f\) fast überall und 
\(f(X)\subseteq\mdr\).
Es ist \(f_{n_m}=\sum^{m-1}_{j=1}g_j + f_{n_1}\) und somit 
\[\lvert f_{n_m}\rvert = \lvert f_{n_1}\rvert + \sum^{m-1}_{j=1}g_j \leq \lvert f_{n_1}\rvert +
\lvert g\rvert\]
Wie im Beweis von Satz \ref{Satz 16.1} folgern wir
\[\lvert f_{n_m}\rvert^p\leq 2^p\left(\lvert f_{n_1}\rvert^p+g^p\right)=:\tilde g \]
 \(f_{n_1}\in\fl^p(X)\), \(g^p\) ist integrierbar. Aus \ref{Satz 16.3} folgt, dass \(f\in\fl^p(X)\)
und \[\| f_{n_m}-f\|_p\to 0 \ (m\to\infty)\]
Sei nun \(\ep>0\). Wähle \(m\in M\) so, dass \(\frac1{2^m}<\frac\ep2\) und 
\(\| f-f_{n_m}\|_p<\frac\ep2\).
Für \(l\geq n_m\) gilt: 
\[\| f_l-f\|_p= \| f_l-f_{n_m}+f_{n_m}-f\|_p
\leq \| f_l-f_{n_m}\|_p + \| f_{n_m}-f\|_p
\overset{\text{(+)}}< \frac1{2^m}+\frac\ep2 <\ep\]
Das heißt
\[\| f_l-f\|_p\to0 \ (l\to\infty)\]
\end{beweis}

\begin{satz}
\label{Satz 16.5}
Sei auch noch \(1\leq q<\infty\). \((f_n)\) sei eine Folge in \(\fl^p(X)\cap\fl^q(X)\). Es sei
\begin{align*}
f\in\fl^p(X) & \text{ und } g\in\fl^q(X)
\intertext{Weiter gelte: }
\| f_n-f\|_p\to 0 & \text{ und } \| f_n-g\|_q\to 0 \ (n\to\infty)
\end{align*}
Dann ist fast überall \(f=g\).
\end{satz}

\begin{beweis}
\begin{enumerate}
\item[\textbf{1.}]
	Aus Bemerkung (3) vor \ref{Satz 16.4} folgt, dass \((\hat f_n)\) ist eine Cachyfolge in
	\(L^p(X)\). Wegen \ref{Satz 16.4} existiert dann ein \(\varphi\in\fl^p(X)\) und eine Teilfolge
	\((f_{n_j})\) mit: \(f_{n_j}\to\varphi\) fast überall und 
	\(\| f_n-\varphi\|_p\to0\)
	\begin{align*}
		\| f-\varphi\|_p
		= \| f-f_n+f_n-\varphi\|_p
		\leq \| f-f_n\|_p + \| f_n-\varphi\|_p
		\to 0\ \ (n\to\infty)
	\end{align*}
	Somit ist \(\| f-\varphi\|_p=0\) und deshalb fast überall \(f=\varphi\).
	Also gilt fast überall \(f_{n_j}\to f\). Das heißt, dass es eine Nullmenge \(N_1\subseteq X\) gibt,
	für die gilt: \[f_{n_j}(x)\to f(x) \text{ für alle } x\in X\setminus N_1\]
\item[\textbf{2.}]
	Setze \(g_j:=f_{n_j}\), dann gilt \(\| g_j-g\|_q\to0\ \ (j\to\infty)\). Wie
	im ersten Schritt zeigt man, dass eine Nullmenge \(N_2\subseteq X\) und eine Teilmenge
	\((g_{j_k})\) existiert mit, für die gilt:
	\[g_{j_k}(x)\to g(x) \text{ für alle } x\in X\setminus N_2\]
\end{enumerate}
Wir wissen, dass \(N:=N_1\cup N_2\) eine Nullmenge ist. Sei nun \(x\in X\setminus N\). Dann
folgt aus dem ersten Schritt \(f_{n_j}(x)\to f(x)\) und daraus 
\[ \underbrace{f_{n_{j_k}}(x)}_{=g_{n_{j_k}}(x)}\to f(x) \]
Aus dem Zweiten Schritt folgt dann, dass \(f_{n_{j_k}}(x)\to g(x)\) und somit \(f(x)=g(x)\).
\end{beweis}

\begin{bemerkung}
Seien \(f_n,f\in\fl^p(X)\) und es gelte \(\| f_n-f\|_p\to 0\ \ (n\to\infty)\). Der
Beweis von \ref{Satz 16.5} zeigt, dass eine Teilfolge \((f_{n_j})\) von \((f_n)\) existiert mit 
\(f_{n_j}\to f\) fast überall.
\end{bemerkung}

\begin{bemerkung}
Konvergenz im Sinne der Norm \(\|\cdot\|_p\) und punktweise Konvergenz fast
überall haben im Allgemeinen \textbf{nichts} miteinander zu tun!
\end{bemerkung}

\begin{beispiel}
Sei \((f_n)\) wie im Beispiel vor \ref{Satz 16.4}. Also \(\| f_n-0\|_p\to 0\), aber
\(f_n\nrightarrow 0\) fast überall.
\end{beispiel}

\begin{beispiel}
%Bild einfügen
Sei \(X=[0,1]\) und \(f_n\) sei wie im Bild. \(f_n\) ist stetig, also messbar. 
\[\int_X f_n\,dx=1 \text{ für alle } \natn\]
Somit ist \(f_n\in\fl^1(X)\).
\[f_n(x)\to
\begin{cases}
0, x\in(0,1]\\
1, x=0
\end{cases}\]
Damit gilt fast überall \(f_n\to0\), aber 
\(\| f_n-0\|_1=1\nrightarrow0 \ \ (n\to\infty)\)
\end{beispiel}

\begin{definition}
	\index{Reihe ! unendliche}
	\index{stetig}
Seien \((E,\|\cdot\|_1), (F,\|\cdot\|_2)\) normierte Räume.
\begin{enumerate}
\item 	Sei \((x_n)\) eine Folge in $E$ und \(s_n:=x_1+x_2+\cdots+x_n\) (\natn).
 		Dann heißt \((s_n)\) eine \textbf{unendliche Reihe} und wird mit
		\[\sum^\infty_{n=1}x_n\] bezeichnet. \(\sum^\infty_{n=1}x_n\) heißt
		\textbf{konvergent} genau dann, wenn \((s_n)\) konvergiert. In diesem Fall ist
		\[\sum^\infty_{n=1}x_n:=\lim_{n\to\infty}s_n\]
\item 	\(\Phi\colon E\to F\) sei eine Abbildung. \(\Phi\) heißt \textbf{stetig} in \(x_0\in E\)
		genau dann, wenn für jede konvergente Folge \((x_n)\) in $E$ mit \(x_n\to x_0\)
		gilt: \[\Phi(x_n)\to\Phi(x_0)\] 
		\(\Phi\) heißt auf $E$ stetig genau dann, wenn \(\Phi\) ist in jedem \(x\in E\) stetig.
\item Für $(x,y)\in E\times E$ setze 
\[\|(x,y)\|:=\sqrt{\|x\|_1^2+\|y\|_1^2}\]
Dann ist $\|\cdot\|$ eine Norm auf $E\times E$ (nachrechnen!). Weiter gilt, dass $E\times E$ genau dann ein Banachraum ist, wenn $E$ einer ist. Für eine Folge $((x_n,y_n))$ in $E\times E$ und $(x,y)\in E\times E$ gilt
\[(x_n,y_n)\stackrel{\|\cdot\|}\to (x,y) \iff x_n\stackrel{\|\cdot\|}\to x \wedge y_n\stackrel{\|\cdot\|}\to y\]
\end{enumerate}
\end{definition}

\begin{bemerkung}
Ist $(x_n)$ eine konvergente Folge in $E$, so ist $(x_n)$ beschränkt (d.h. $\exists c>0: \|x_n\|_1\le c \forall n\in\mdn$).

(Beweis wie in Ana I)
\end{bemerkung}

\begin{vereinbarung}
Für den Rest dieser Vorlesung schreiben wir (meist) $f$ statt $\hat f$ und identifizieren $\fl^p(X)$ mit $L^p(X)$. Ebenso schreiben wir $\int_X f\text{ d}x$ statt $\int_X \hat f\text{ d}x$ und $(f|g)$ statt $(\hat f|\hat g)$.
\end{vereinbarung}

\begin{wichtigesbeispiel}
\label{Beispiel 16.6}
\begin{enumerate}
\item Die Abbildung $\Phi:L^p(X)\to\mdr$, definiert durch
\[\Phi(f):=\|f\|_p\]
ist stetig auf $L^p(X)$. D.h. für $f_n,f\in L^p(X)$ mit $f_n\stackrel{\|\cdot\|_p}\to f$ gilt $\|f_n\|_p\to\|f\|_p$, also
\[\int_X|f_n|^p\text{ d}x\to\int_X|f|^p\text{ d}x\] 
\begin{beweis}
Aus Analysis II §17 folgt:
\[| \|f_n\|_p-\|f\|_p |\le \|f_n-f\|_p\stackrel{n\to\infty}\to 0\]
\end{beweis}
\item Die Abbildung $\Phi:L^1(X)\to\mdr$ definiert durch
\[\Phi(f):=\int_X f\text{ d}x\]
ist stetig auf $L^1(X)$. D.h. aus $f_n,f\in L^1(X)$ und $f_n\stackrel{\|\cdot\|_1}\to f$ folgt
\[\int_X f_n\text{ d}x\to\int_X f \text{ d}x\]
\begin{beweis}
Es gilt:
\begin{align*}
|\int_X f_n \text{ d}x-\int_X f \text{ d}x| &=|\int_X f_n-f \text{ d}x|\\
&\le \int_X |f_n-f| \text{ d}x\\
&= \|f_n-f\|_1\stackrel{n\to\infty}\to 0
\end{align*}
\end{beweis}
\item Die Abbildung $\Phi:L^2(X)\times L^2(X)\to\mdr$ definiert durch
\[\Phi(f,g):=(f|g)\]
ist stetig auf $L^2(X)\times L^2(X)$. D.h. für $f_n,g_n,f,g\in L^2(X)$ mit $f_n\stackrel{\|\cdot\|_2}\to f$ und $g_n\stackrel{\|\cdot\|_2}\to g$ gilt
\[(f_n|g_n)\stackrel{n\to\infty}\to(f|g)\]
\begin{beweis}
Es gilt:
\begin{align*}
|(f_n|g_n)-(f|g)|&=|(f_n|g_n)-(f_n|g)+(f_n|g)-(f|g)|\\
&=|(f_n|g_n-g)+(f_n-f|g)|\\
&\le |(f_n|g_n-g)|+|(f_n-f|g)|\\
&\le \|f_n\|_2\cdot \|g_n-g\|_2 + \|f_n-f\|_2\cdot\|g\|_2\stackrel{n\to\infty}\to 0
\end{align*}
\end{beweis}
\end{enumerate}
\end{wichtigesbeispiel}

\begin{satz}
\label{Satz 16.7}
Sei $f=f_+-f_-\in L^p(X)$ und $(g_n)$ und $(h_n)$ seien zulässige Folgen für $f_+$ bzw. $f_-$ (d.h. $g_n,h_n$ einfach, $0\le g_n\le g_{n+1}, g_n\to f_+$, $0\le h_n\le h_{n+1}, h_n\to f_-$). Setze $f_n:=g_n-h_n$.\\
Dann sind $f_n,g_n,h_n\in L^p(X)$ und es gilt:
\begin{align*}
&\|g_n-f_+\|_p\to 0&&\|h_n-f_-\|_p\to 0&&\|f_n-f\|_p\to 0
\end{align*}
\end{satz}

\begin{beweis}
Es genügt den Fall $f\ge 0$ zu betrachten (also $f=f_+$, $f_-\equiv 0$). Sei also $(f_n)$ zulässig für $f$. Definiere $\varphi:=|f_n-f|^p$. Es ist klar, dass punktweise gilt $\varphi_n\to 0$. Außerdem gilt:
\begin{align*}
0\le\varphi_n&\le (|f_n|+|f|)^p\\
&=|f_n+f|^p\le (2f)^p\\
&=2^pf^p=:g 
\end{align*}
Dann ist $g\in L^1(X)$ integrierbar.\\
Aus \ref{Satz 4.9} folgt:
\begin{align*}
\varphi\in L^1(X)&\implies f_n-f\in L^p(X)\\
&\implies f_n=(f_n-f)+f\in L^p(X) 
\end{align*}
Aus \ref{Satz 6.2} folgt:
\[\int_X\varphi_n\text{ d}x\to 0 \implies \|f_n-f\|_p^p\to 0\]
\end{beweis}

\begin{definition}
\index{Träger}
\begin{enumerate}
\item Sei $f:X\to\mdr$. Dann heißt
\[\supp (f):=\overline{\{x\in X\mid f(x)\ne 0\}}\]
der \textbf{Träger} von $f$
\item $C_c(X,\mdr):=\{f\in C(X,\mdr)\mid \supp(f)\subseteq X\text{ und } \supp(f) \text{ kompakt}\}$
\end{enumerate}
\end{definition}

\begin{satz}
\index{dicht}
\label{Satz 16.8}
\begin{enumerate}
\item $C_c(X,\mdr)\subseteq L^p(X)$
\item Ist $X$ offen, so liegt $C_c(X,\mdr)$ \textbf{dicht} in $L^p(X)$, d.h. ist $f\in L^p(X)$ und $\ep>0$, so existiert $g\in C_c(X,\mdr)$ mit $\|f-g\|_p<\ep$.
\end{enumerate}
\end{satz}

\begin{beweis}
\begin{enumerate}
\item Sei $f\in C_c(C,\mdr)$ und $K:=\supp(f)$, dann ist $K\subseteq X$ kompakt, also $K\in\fb_d$. Es gilt für alle $x\in X\setminus K$ $f(x)=0$ und damit folgt aus \ref{Satz 4.12} $\int_K |f|^p\text{ d}x<\infty$. Dann gilt:
\[\int_X |f|^p\text{ d}x=\int_{X\setminus K} |f|^p\text{ d}x+\int_K |f|^p\text{ d}x=\int_K |f|^p\text{ d}x<\infty\]
Also ist $f\in L^p(X)$.
\item Siehe Übungsblatt 13.
\end{enumerate}
\end{beweis}

\chapter{Das Integral im Komplexen}
\label{Kapitel 17}



Definitionen:

\begin{longtable}{ m{2.5cm} m{3cm} m{5cm} m{6cm} }
  \textbf{Name} & Symbol & Definition & in Worten/Kommentar \\ 
  \hline 
  	& $ $ 
  	& $ $ 
  	& \\
\end{longtable}


In diesem Paragraphen sei $\varnothing \ne X \in \fb_d, f: X \to \MdC$ eine Funktion, $ u:= \Re(f), v:= \Im(f)$, also: $u,v: X \to \MdR, f= u+iv$.

Wir versehen $\MdC$ mit der $\sigma$-Algebra $\fb_2$ (wir identifizieren $\MdC$ mit $\mdr^2$).

\begin{definition}
\index{messbar}
$f$ hei"st (Borel-)\textbf{messbar}, genau dann wenn gilt: $f$ ist $\fb_d$-$\fb_2$-messbar.
\end{definition}

Aus 3.2 folgt: $f$ ist messbar genau dann, wenn $u$ und $v$ messbar sind.

\begin{definition}
\index{integrierbar}\index{Integral}
Sei $f$ messbar. $f$ hei"st \textbf{integrierbar} (ib.) genau dann, wenn $u$ und $v$ integrierbar sind.
In diesem Fall setze 
\[ \int_X f \text{ d}x := \int_X u \text{ d}x + i\int_X v \text{ d}x \quad ( \in \MdC) \]
\end{definition}

Es gilt: $|u|, |v| \leq |f| \leq |u| + |v|$ auf $X$.
Hieraus und aus 4.9 folgt: $f$ ist integrierbar genau dann, wenn $|f|$ integrierbar ist.

\begin{definition}
\[ \fl^p(X, \MdC) := \{ f : X \to \MdC | f  \text{ ist messbar und } \int_X |f|^p \text{ d}x < \infty  \} \]
(Achtung: mit den Betragsstrichen in ob. Integral ist der komplexe Betrag gemeint!)
\[ \cn := \{ f: X \to \MdC | f \text{ ist messbar und } f = 0 \text{ f.\"u.} \} \]
$\fl^p(X,\MdC )$ ist ein komplexer Vektorraum (siehe 17.1) und $\cn$ ist ein Untervektorraum von $\fl^p(X,\MdC )$. 
\[ L^p(X,\MdC ) := \fl^p(X,\MdC)\diagup\cn \]
\end{definition}

\begin{definition}
\index{orthogonal}
F\"ur $f,g \in L^2(X,\MdC )$ setze 
\[(f | g) := \int_X f(x) \overline{g(x)} \text{ d}x\]
sowie 
\[f \bot g :\Longleftrightarrow (f | g) = 0 \quad \text{ ($f$ und $g$ sind \textbf{orthogonal}).} \]
( $\overline{z}$ bezeichne hierbei die komplex Konjugierte von $z$, vgl. Lineare Algebra).
\end{definition}

\textbf{Klar:} \begin{enumerate}
\item $L^p(X,\MdC )$ ist mit $\| f \|_p := (\int_X |f|^p \text{ d}x )^{\frac{1}{p}}$ ein komplexer normierter Raum (NR).
\item $(f | g)$ definiert ein Skalarprodukt auf $L^2(X,\MdC)$. Es ist 
\[(f | g) = \overline{(g | f)}, \]
\[ (f | f) = \int_X f(x) \overline{f(x)} \text{ d}x = \int_X |f(x)|^2 \text{ d}x = \| f \|_2^2 \text{, also:} \]
\[ \| f\|_2 = \sqrt{(f|f)} \quad (f,g \in L^2(X,\MdC )) \]
(Beachte: es ist $z \cdot \overline{z} = |z|^2$ f\"ur $z \in \MdC$).
\end{enumerate}

\textbf{Inoffizielle Anmerkung:} Dieses Skalarprodukt ist auf $\MdC$ nur linear in der ersten Komponente! Wenn man einen $\MdC$-Skalar aus der zweiten Komponente rausziehen m\"ochte, muss man diesen komplex konjugieren:
\begin{align*}
\alpha \in \MdC:\quad &(f|\alpha g) = \overline{\alpha} (f|g)\\
&(\alpha f|g) = \alpha (f | g)
\end{align*}

\begin{satz}
\label{Satz 17.1}
\begin{enumerate}
	\item 	Seien \(f,g\colon X\to\mdc\) integrierbar und \(\alpha,\beta\in\mdc\). Dann gelten:
	\begin{enumerate}
		\item[(i)] 	\(\alpha f+\beta g\) ist integrierbar und 
				\[\int_X(\alpha f+\beta g)\,dx = \alpha\int_Xf\,dx+\beta\int_Xg\,dx\]
		\item[(ii)]	\(\text{Re}\left(\int_Xf\,dx\right) = \int_X\text{Re}(f)\,dx\ \) und
				\(\ \text{Im}\left(\int_Xf\,dx\right) = \int_X\text{Im}(f)\,dx\)
		\item[(iii)]	\(\overline f\) ist integrierbar und 
				\[\int_X\overline f\,dx=\overline{\int_Xf\,dx}\]
	\end{enumerate}
	\item 	Die Sätze \ref{Satz 16.1} bis \ref{Satz 16.3} und das Beispiel \ref{Beispiel 16.6} gelten in 
			\(L^p(X,\mdc)\).
	\item 	\(L^p(X,\mdc)\) ist ein komplexer Banachraum, \(L^2(X,\mdc)\) ist ein komplexer 
			Hilbertraum.
\end{enumerate}
\end{satz}

\begin{wichtigesbeispiel}
\label{Beispiel 17.2}
Sei \(X=[0,2\pi]\). Für \(k\in\MdZ\) und \(t\in\mdr\) setzen wir
\begin{align*}
	e_k(t):=e^{ikt}=\cos(kt)+i\sin(kt) && \text{ und } && b_k:=\frac1{\sqrt{2\pi}}e_k
\end{align*}
Dann gilt: \(b_k,e_k\in L^2([0,2\pi],\mdc)\) und \[\int_0^{2\pi}e_0(t)\,dt=2\pi\]
Für \(k\in\MdZ\) und \(k\neq0\) ist
\begin{align*}
	\int_0^{2\pi}e_k(t)\,dt=\left.\frac1{ik}e^{ikt}\right\rvert_0^{2\pi} 
	= \frac1{ik}\left(e^{2\pi ki}-1\right)=0
\intertext{Damit ist}
	(b_k\mid b_l) = \int^{2\pi}_0 b_k\overline{b_l}\,dt = \frac1{2\pi}\int_0^{2\pi}e^{ikt}e^{-ilt}\,dt
	= \frac1{2\pi}\int_0^{2\pi}e^{i(k-l)t}\,dt =
	\begin{cases} 
		1 ,\text{falls } k=l\\
		0 ,\text{falls }k\neq l
	\end{cases}
\end{align*}
Insbesondere ist \(\| b_k\|_2=1\). Das heißt \(\{b_k\mid k\in\MdZ\}\) ist ein
\textbf{Orthonormalsystem} in \(L^2([0,2\pi],\mdc)\).
Zur Übung: \(\{b_k\mid k\in\MdZ\}\) ist linear unabhängig in \(L^2([0,2\pi],\mdc)\).
\end{wichtigesbeispiel}

\begin{definition}
Sei \((\alpha_k)_{k\in\MdZ}\) eine Folge in \(\mdc\) und \((f_k)_{k\in\MdZ}\) eine Folge in 
\(L^2(X,\mdc)\).
\begin{enumerate}
	\item 	Für \(n\in\mdn_0\) setze 
			\[s_n:=\sum^n_{k=-n}\alpha_k = \sum_{\lvert k\rvert\leq n}\alpha_k
			=\alpha_{-n}+\alpha_{-(n-1)}+\cdots+\alpha_0+\alpha_1+\cdots+\alpha_n\]
			Existiert \(\lim_{n\to\infty}s_n\) in \(\mdc\), so schreiben wir
			\(\sum_{k\in\MdZ}\alpha_k:=\lim_{n\to\infty}s_n\)
	\item 	Für \(n\in\mdn_0\) setze 
			\[\sigma_n:=\sum^n_{k=-n}f_k=\sum_{\lvert k\rvert\leq n}f_k\]
			Gilt für ein \(f\in L^2(X,\mdc)\): 
			\(\| f-\sigma_n\|_2\overset{n\to\infty}\longrightarrow 0\), so schreiben
			wir \[f\overset{\|\cdot\|_2}=\sum_{k\in\MdZ}f_k \ \ \ 
			\left(=\lim_{n\to\infty}\sigma_n \text{ im Sinne der } L^2\text{-Norm}\right)\]
\end{enumerate}
\end{definition}

\begin{definition}
\index{Orthonormalbasis}
Sei \(\{b_k\mid k\in\MdZ\}\) wie in \ref{Beispiel 17.2}. \(\{b_k\mid k\in\MdZ\}\) heißt eine
\textbf{Orthonormalbasis (ONB)} von \(L^2([0,2\pi],\mdc)\) genau dann, wenn es zu jedem
\(f\in L^2([0,2\pi],\mdc)\) eine Folge \[(c_k)_{k\in\MdZ}=(c_k(f))_{k\in\MdZ}\] gibt, mit
\[(\ast)\ \ \ \ \ \ \ \ \ f\overset{\|\cdot\|_2}=\sum_{k\in\MdZ}c_kb_k \]
\textbf{Frage:} Ist \(\{b_k\mid k\in\MdZ\}\) eine ONB von \(L^2([0,2\pi],\mdc)\)?\\
\textbf{Antwort:} Ja! In \ref{Satz 18.5} werden wir sehen, dass \((\ast)\) gilt mit 
\(c_k=(f\mid b_k)\).
\end{definition}

\chapter{Fourierreihen}
\label{Kapitel 18}



Definitionen:

\begin{longtable}{ m{2.5cm} m{3cm} m{5cm} m{6cm} }
  \textbf{Name} & Symbol & Definition & in Worten/Kommentar \\ 
  \hline 
  	& $ $ 
  	& $ $ 
  	& \\
\end{longtable}


In diesem Paragraphen sei stets \(X=[0,2\pi]\), \(L^2:=L^2([0,2\pi],\mdc)\) und 
\(L^2_\mdr:=L^2([0,2\pi],\mdr)\).  Weiter sei \(\{b_k\mid k\in\MdZ\}\) wie in \ref{Beispiel 17.2}.

\begin{satz}
\label{Satz 18.1}
Ist \(f\in L^2\) und gilt mit einer Folge \((c_k)_{k\in\MdZ}\) in \(\mdc\): 
\(f\overset{\|\cdot\|_2}=\sum_{k\in\MdZ}c_kb_k \), so gilt:
\[c_k=(f\mid b_k) \text{ für alle } k\in\MdZ\]
\end{satz}

\begin{beweis}
Für \(n\in\mdn_0\) setze \[\sigma_n:=\sum_{\lvert k\rvert\leq n}c_kb_k\] Aus der Voraussetzung folgt
\(\| \sigma_n-f\|_2\to 0\) für \(n\to\infty\). Sei \(j\in\MdZ\) und \(n\in\mdn\) mit
\(n\geq \lvert j\rvert\). Es gilt einerseits 
\[(\sigma_n\mid b_j) = \sum_{\lvert k\rvert\leq n}c_k(b_k\mid b_j)=c_j, \text{ da gilt: }
(b_k\mid b_j)=
\begin{cases}
0, \text{ falls } k\neq j\\
1, \text{ falls } k= j
\end{cases}\]
Andererseits: \((\sigma_n\mid b_j)\to(f\mid b_j)\) für \(n\to\infty\) wegen \ref{Beispiel 16.6}(3). Daraus
folgt \(c_j=(f\mid b_j)\)
\end{beweis}

\begin{definition}
\index{Fourier ! -sche Partialsumme}
\index{Fourier ! -koeffizient}
\index{Fourier ! -reihe}
Sei \(f\in L^2\), \(n\in\mdn_0\) und \(k\in\MdZ\).
\begin{enumerate}
\item	\(S_nf:=\sum_{\lvert k\rvert\leq n}(f\mid b_k)b_k\) heißt
	\textbf{n-te Fouriersche Partialsumme}. Also gilt:
	\[f\overset{\|\cdot\|_2}
	=\sum_{k\in\MdZ}(f\mid b_k)b_k\gdw\| f-S_nf\|_2
	\to0\]
\item	\((f\mid b_k)\) heißt \textbf{k-ter Fourierkoeffizient von f}.
\item	\(\sum_{k\in\MdZ}(f\mid b_k)b_k\) heißt \textbf{Fourierreihe von f}.
\item	Für \(n_0\in\mdn_0\) setze 
	\(E_n:=[b_{-n},b_{-(n-1)},\ldots,b_0,b_1,\ldots,b_n]\) 
	(lineare Hülle). Es ist dann \[\dim E_n=2n+1\]
	\textbf{Beachte: } Für \(v\in E_n\) gilt \(v(0)=v(2\pi)\).
\end{enumerate} 
\end{definition}

\begin{satz}
\label{Satz 18.2}
\index{Besselsche Ungleichung}
\index{Ungleichung ! Besselsche}
Seien \(f_1,\ldots,f_n,f\in L^2\).
\begin{enumerate}
\item	Gilt \(f_\mu\perp f_\nu\) für \(\mu\neq\nu\) (\(\mu,\nu=1,\ldots,n\)),
	so gilt der Satz des Pythagoras 
	\[\| f_1+\cdots+f_n\|^2_2=
	\| f_1\|^2_2+\cdots+
	\| f_n\|^2_2\]
\item	Die Abbildung \[S_n\colon
	\begin{cases}
		L^2\to E_n\\
		S_nf:=\sum_{\lvert k\rvert\leq n}(f\mid b_k)b_k
	\end{cases}\]
	ist linear und für jedes \(v\in E_n\) gilt \(S_nv=v\) und 
	\((f-S_nf)\perp v\) mit \(f\in L^2\).
\item 	Die \textbf{Besselsche Ungleichung} lautet:
	\[\| S_nf\|^2_2
	=\sum_{\lvert k\rvert\leq n}\lvert(f\mid b_k)\rvert^2
	=\| f\|_2^2-\|(f-S_nf)\|^2_2
	\leq\| f\|^2_2\]
\item	Für alle \(v\in E_n\) gilt: 
	\[\| f-S_nf\|_2\leq\| f-v\|_2 
	\]
\end{enumerate}
\end{satz}

\begin{beweis}
\begin{enumerate}
\item	Es genügt den Fall \(n=2\) zu betrachten, der Rest folgt induktiv.
	\begin{align*}
	\| f_1+f_2\|_2^2
	&= (f_1+f_2\mid f_1+f_2)						\\
	&= (f_1\mid f_1)+(f_1\mid f_2)+(f_2\mid f_1)+(f_2\mid f_2)		\\
	&= (f_1\mid f_1)+(f_2\mid f_2)						\\
	&=\| f_1\|^2_2+\| f_2\|^2_2	
	\end{align*}
\item	Übung!
\item	Es gilt
	\begin{align*}
	\| S_nf\|^2_2
	&= \left\lvert\left\lvert\sum_{\lvert k\rvert\leq n}(f\mid b_k)b_k\right\rvert
		\right\rvert^2_2	
	\overset{(1)}=
		\sum_{\lvert k\rvert\leq n}\|(f\mid b_k)b_k\rvert
		\rvert^2_2	
	= \sum_{\lvert k\rvert\leq n}\lvert(f\mid b_k)\rvert^2\| b_k\rvert
		\rvert^2_2	
	= \sum_{\lvert k\rvert\leq n}\lvert(f\mid b_k)\rvert^2
	\end{align*}
	und
	\begin{align*}
	\| f\|^2_2 
	= \|\underbrace{(f-S_nf)}_{\underset{(2)}\perp E_n}
		+\underbrace{S_nf}_{\in E_n}\|^2_2		
	= \| f-S_nf\|^2_2 + \| S_nf\|^2_2
	\end{align*}
\item	Sei \(v\in E_n\). Dann gilt:
	\begin{align*}
	\| f-v\|^2_2
	&= \|\underbrace{(f-S_nf)}_{\perp E_n}
		+\underbrace{(S_nf-v)}_{\in E_n}\|^2_2		\\
	&\overset{(1)}=
		\| f-S_nf\|^2_2
		+\| S_nf-v\|^2_2				\\
	&\geq \| f-S_nf\|^2_2
	\end{align*}
\end{enumerate}
\end{beweis}

\begin{wichtigebemerkung}
\label{Bemerkung 18.3}
Es sei \(\mdk\in\{\mdr,\mdc\},\,a,b\in\mdr,\,I:=[a,b]\,(a<b)\) und \(f_{n},\,f,\,g\in C(I,\mdk)\); es war 
\(\lVert f\rVert_{\infty}:=\max_{t\in I}\lvert f(t)\rvert\).
\begin{enumerate}
\item \((f_{n})\) konvergiert auf \(I\) gleichm\"a\ss ig gegen \(f\) genau dann, wenn 
    \(\lVert f_{n}-f\rVert_{\infty}\to 0\,(n\to\infty)\) (vgl. Analysis I/II).
\item \(f\in\mathrm{L}^{p}(I,\mdk)\) und \(\lVert f\rVert_{p}\leq(b-a)^{\frac{1}{p}}\lVert f\rVert_{\infty}\) (siehe \ref{Satz 16.2}).
\item Gilt \(f=g\) fast \"uberall, so ist \(f=g\) auf \(I\).
\begin{beweis}
Es existiert eine Nullmenge \(N\subseteq I:\,f(x)=g(x)\,\forall x\in I\setminus N\).\\
Sei \(x_{0}\in\mdn\). F\"ur \(\ep>0\) gilt: \(U_{\ep}(x_{0})\cap I\not\subseteq N\) (andernfalls: 
    \(\lambda_{1}(N)\geq\lambda_{1}(U_{\ep}(x_{0})\cap I)>0\)). Das hei\ss t, es existiert ein 
    \(x_{\ep}\in U_{\ep}(x_{0})\cap I:\,x_{\ep}\not\in N\). Also:
    \(\forall n\in\mdn\,\exists x_{n}\in U_{\frac{1}{n}}(x_{0})\cap I:\, x_{n}\not\in N\). Also: \(x_{n}\to x_{0}\).\\
Dann: \(f(x_{0})=\lim_{n\to\infty}f(x_{n})=\lim_{n\to\infty}g(x_{n})=g(x_{0})\)
\end{beweis}
\end{enumerate}
\end{wichtigebemerkung}

\begin{satz}[Approximationssatz von Weierstra\ss]
\label{Satz 18.4}
Es sei \(I=[a,b]\) wie in \ref{Bemerkung 18.3} und \(\mdk\in\{\mdr, \mdc\}\).
\begin{enumerate}
\item Ist \(f\in C(I,\mdk)\) und \(\ep>0\), so existiert ein Polynom \(p\) mit Koeffizienten in \(\mdk\) mit:
\[
\lVert f-p\rVert_{\infty}<\ep
\]
\item Ist \(a=0,\,b=2\pi,\,f\in C(I,\mdk),\,f(0)=f(2\pi)\) und \(\ep>0\), so existiert ein \(n\in\mdn\) und ein
    \(v\in\mathrm{E}_{n}\) mit:
\[
\lVert f-v\rVert_{\infty}<\ep
\]
\end{enumerate}
\end{satz}

\begin{satz}
\label{Satz 18.5}
Sei \(f\in\mathrm{L}^{2}\). Dann gilt: \(f\overset{\lVert\cdot\rVert_{2}}{=}\sum_{k\in\mdz}{(f\mid b_{k})b_{k}}\) und
\[\lVert f\rVert_{2}^{2}=\sum_{k\in\mdz}{\lvert(f\mid b_{k})\rvert^{2}}\quad\text{(\textbf{Parsevalsche Gleichung})}\] Insbesondere gilt:
\((f\mid b_{k})\to 0\quad(\lvert k\rvert\to\infty)\).
\end{satz}

\begin{beweis}
Zu zeigen: \(\lVert f-S_{n}f\rVert_{2}\to0\,(n\to\infty)\). Die Parsevalsche Gleichung folgt dann aus \ref{Satz 18.2}.\\
Sei \(\ep>0\). Wende \ref{Satz 16.8}(2) auf \(\Re f\) und \(\Im f\) an. Dies liefert eine stetige Funktion 
\(g:\,(0,2\pi)\to\mdc\) mit: \(K:=\supp(g)\subseteq(0,2\pi)\), \(K\) kompakt und \(\lVert f-g\rVert_{2}<\ep\).\\
Setze \(g(0):=g(2\pi):=0\). Dann ist \(g\) stetig auf \([0,2\pi]\). Satz \ref{Satz 18.4} liefert nun: 
    \(\exists n\in\mdn\exists v\in\mathrm{E}_{n}:\,\lVert g-v\rVert_{\infty}<\ep\).\\
Damit: \(\lVert g-v\rVert_{2}\leq\sqrt{2\pi}\lVert g-v\rVert_{\infty}<\sqrt{2\pi}\ep\). Somit:
\begin{align*}
\lVert f-S_{n}f\rVert_{2}&=\lVert f-g+g-S_{n}g+S_{n}g-S_{n}f\rVert_{2}\\
    &\leq\underbrace{\lVert f-g\rVert_{2}}_{<\ep}
		+\underbrace{\lVert g-S_{n}g\rVert_{2}}_{\overset{18.2(4)}{\leq}\lVert g-v\lVert_2}
		+\underbrace{\lVert S_{n}(g-f)\rVert_{2}}_{\overset{18.2(3)}{\leq}\lVert g-f\lVert_2}\\
    &<2\ep+\sqrt{2\pi}\ep=\ep(2+\sqrt{2\pi})
\end{align*}
Sei \(m\geq n\). Dann gilt: \(\mathrm{E}_{n}\subseteq\mathrm{E}_{m}\), also \(w:=S_{n}f\in\mathrm{E}_{m}\). Damit:
\[
\lVert f-S_{m}f\rVert_{2}\leq\lVert f-w\rVert_{2}=\lVert f-S_{n}f\rVert_{2}<\ep(2+\sqrt{2\pi})
\]
\end{beweis}

\subsubsection*{Reelle Version}
Sei \(f\in\mathrm{L}^{2}_{\mdr}\).
Es gelten die folgenden Bezeichnungen:
\begin{enumerate}
\item F\"ur \(k\in\mdn\) bezeichnen wir die Funktionen \(t\mapsto\cos(kt)\) und \(t\mapsto\sin(kt)\) mit \(\cos(k\cdot)\) bzw.
    \(\sin(k\cdot)\).
\item F\"ur \(k\in\mdn_{0}:\,\alpha_{k}:=\frac{1}{\pi}\int_{0}^{2\pi}{f(t)\cos(kt)\mathrm{d}t}=\frac{1}{\pi}\Re(f\mid e_{k})\).\\
F\"ur \(k\in\mdn:\,\beta_{k}:=\frac{1}{\pi}\int_{0}^{2\pi}{f(t)\sin(kt)\mathrm{d}t}=\frac{1}{\pi}\Im(f\mid e_{k}),\,\beta_{0}:=0\).
\end{enumerate}

\begin{definition}
\index{gerade Funktion}
\index{ungerade Funktion}
\(f\) hei\ss t \textbf{gerade} (bez\"uglich \(\pi\)) genau dann, wenn gilt: \(f(t)=f(2\pi-t)\) f\"ur fast alle \(t\in[0,2\pi]\).\\
\(f\) hei\ss t \textbf{ungerade} (bez\"uglich \(\pi\)) genau dann, wenn gilt: \(f(t)=-f(2\pi-t)\) f\"ur fast alle \(t\in[0,2\pi]\).\\
% Bild nicht vergessen...
\end{definition}

\begin{satz}
\label{Satz 18.6}
(Dieser Satz folgt aus \ref{Satz 18.5} und ``etwas'' rechnen)\\
Sei \(f\in\mathrm{L}^{2}_{\mdr}\) und \(n\in\mdn_{0}\).
\begin{enumerate}
\item \(S_{n}f=\frac{\alpha_{0}}{2}+\sum_{k=1}^{n}{(\alpha_{k}\cos(k\cdot)+\beta_{k}\sin(k\cdot))}\)
\item \(f\overset{\lVert\cdot\rVert_{2}}{=}\frac{\alpha_{0}}{2}+\sum_{k=1}^{\infty}{(\alpha_{k}\cos(k\cdot)+\beta_{k}\sin(k\cdot))}\)
\item \(\frac{1}{\pi}\lVert f\rVert_{2}^{2}=\frac{\alpha_{0}^{2}}{2}+\sum_{k=1}^{\infty}{(\alpha_{k}^{2}+\beta_{k}^{2})}\quad\)
    (Parsevalsche Gleichung)\\
Insbesondere gilt: \(\alpha_{k}\to0,\,\beta_{k}\to0\quad(k\to\infty)\)
\item Ist \(f\) gerade, so sind alle \(\beta_{k}=0\) und \(\alpha_{k}=\frac{2}{\pi}\int_{0}^{\pi}{f(t)\cos(kt)\mathrm{d}t}\). Die
Fourierreihe von \(f\) ist eine \textbf{Cosinusreihe}.\\
Ist \(f\) ungerade, so sind alle \(\alpha_{k}=0\) und \(\beta_{k}=\frac{2}{\pi}\int_{0}^{\pi}{f(t)\sin(kt)\mathrm{d}t}\). Die
Fourierreihe von \(f\) ist eine \textbf{Sinusreihe}.
\end{enumerate}
\end{satz}

\begin{beispiele}
\begin{enumerate}
\item \(f(t):=\begin{cases}1,&0\leq t\leq\pi\\-1,&\pi<t\leq 2\pi\end{cases}\)

\(f\) ist ungerade, also \(\alpha_{k}=0\,\forall k\in\mdn_{0}\). Es ist 
\(\beta_{k}=\frac{2}{\pi}\int_{0}^{\pi}{\sin(kt)\mathrm{d}t}=\begin{cases}0,&k\text{ gerade}\\\frac{4}{k\pi},&k\text{ ungerade}\end{cases}\).\\
Damit: 
\[
f\overset{\lVert\cdot\rVert_{2}}{=}\frac{4}{\pi}\sum_{j=0}^{\infty}{\frac{\sin((2j+1)\cdot)}{2j+1}}
\]
Beachte: \((S_{n}f)(0)=0\to 0\neq1=f(0)\) und \((S_{n}f)(2\pi)=0\to 0\neq -1=f(2\pi)\).
\item \(f(t):=\begin{cases}t,&0\leq t\leq\pi\\2\pi-t,&\pi\leq t\leq 2\pi\end{cases}\)\\
\(f\) ist gerade, das hei\ss t \(\beta_{k}=0\,\forall k\in\mdn\) und \(\alpha_{k}=\frac{2}{\pi}\int_{0}^{\pi}{t\cos(kt)\mathrm{d}t},\,\alpha_{0}=\pi\).\\
F\"ur \(k\geq 1:\quad\alpha_{k}=\begin{cases}0,&k\text{ gerade}\\-\frac{4}{\pi k^{2}},&k\text{ ungerade}\end{cases}\).\\
Damit:
\[
f\overset{\lVert\cdot\rVert_{2}}{=}\frac{\pi}{2}-\frac{4}{\pi}\sum_{j=0}^{\infty}{\frac{\cos((2j+1)\cdot)}{(2j+1)^{2}}}
\]
\end{enumerate}
\end{beispiele}
% Ende der reellen Version

\begin{satz}
\label{Satz 18.7}
Sei $f \in L^2$ und $\sum_{k \in \MdZ} |(f|b_k)| < \infty$. Dann:
\begin{enumerate}
\item Die Reihe $\sum_{k \in \MdZ} (f\mid b_k) b_k(t)$ konvergiert auf $[0, 2 \pi ]$ absolut und gleichmäßig.
Setzt man $g(t) := \sum_{k \in \MdZ} (f\mid b_k)b_k(t)$ für $t \in [0, 2\pi ]$, so ist $g$ stetig, $g(0)=g(2\pi )$ und $f=g$ f.ü. auf $[0,2 \pi ]$.
\item Ist $f$ stetig, so gilt $f=g$ auf $[0,2\pi ]$, also:
\begin{equation*}
\label{Gleichung 2, Satz 18.7}
f(t)=\sum_{k\in\MdZ}(f\mid b_k)b_k(t)\quad\forall t\in[0,2\pi]
\end{equation*}
Insbesondere: $f(0)=f(2\pi)$
\end{enumerate}
\end{satz}

\begin{beweis}
\begin{enumerate}
\item $f_k(t) := (f\mid b_k)b_k(t)$;
\[
\lvert f_k(t)\rvert=\lvert(f\mid b_k)\rvert\cdot\lvert b_k(t)\rvert=\frac{1}{\sqrt{2\pi}}\lvert(f\mid b_k)\rvert\quad\forall t \in [0,2\pi ] \forall k \in \MdZ
\]
Aus Analysis I, 19.1(2) (Konvergenzkriterium von Weierstraß) folgt: Die Reihe in (1) konvergiert auf $[0,2\pi]$ absolut und gleichmäßig.
Aus Analysis I, 19.2 folgt: $g$ ist stetig.
Klar: $g(0) = g(2\pi )$.
\[ s_n(t) := \sum_{\lvert k\rvert \leq n} f_k(t) \quad (n \in \MdN_0, t \in [0,2\pi ]).\]
Aus \ref{Satz 18.5} folgt: $\| f-s_n \|_2 \to 0 (n\to \infty )$.
$\| g-s_n \|_2 \overset{18.3(2)}{\leq} \| g-s_n \|_\infty \sqrt{2\pi } \to 0 (n\to \infty )$
Also: $\| g -s_n\|_2 \to 0 (n \to \infty)$
Aus \ref{Satz 16.5} folgt: $f=g$ f.ü.
\item $f=g$ f.ü. $\overset{18.3(3)}{\implies}\,f=g$ auf $[0,2\pi]$.
\end{enumerate}
\end{beweis}

\begin{satz}
\label{Satz 18.8}
$f \in L^2_\MdR$ und die Folgen $(\alpha_k )$ und $(\beta_k )$ seien definiert wie im Abschnitt ``Reelle Version''. Weiter gelte: $\sum_{k=1}^\infty\lvert\alpha_k\rvert<\infty$ und $\sum_{k=1}^\infty\lvert\beta_k\rvert<\infty$. Dann gelten die Aussagen in \ref{Satz 18.7} für die Reihen in \ref{Satz 18.6}.
\end{satz}

\begin{satz}
\label{Satz 18.9}
Sei $f:[0,2\pi] \to \MdC$ \textbf{stetig differenzierbar} und $f(0)=f(2\pi)$.
\begin{enumerate}
\item Es ist $(f'\mid b_k)=ik(f\mid b_k)\quad\forall k\in\MdZ$
\item $\sum_{k\in\MdZ}\lvert(f\mid b_k)\rvert<\infty$  (d.h.: die Voraussetzungen von \ref{Satz 18.7} sind erfüllt)
\end{enumerate}
\end{satz}

\begin{beweis}
\begin{enumerate}
\item \begin{align*}
(f'|b_k) &= \frac{1}{\sqrt{2\pi}} \int_0^{2\pi} f'(t)e^{-ikt} \text{ d}t  \\
&\overset{P.I.}{=} \frac{1}{\sqrt{2\pi}} \left[ f(t)e^{-ikt} \right]_0^{2\pi} - \frac{1}{\sqrt{2\pi}}\int_0^{2\pi} f(t)(-ik)e^{-ikt}\text{ d}t \\
&= \frac{1}{\sqrt{2\pi}}(f(2\pi ) - f(0)) + ik(f|b_k).
\end{align*}
\item Setze $\sigma_n := \sum_{|k|\leq n} |(f|b_k)| \quad (n \in \MdN_0)$. Es genügt zu zeigen: $(\sigma_n )$ ist beschränkt. Klar: $0 \leq \sigma_n$.
\begin{align*}
\sigma_n - |(f|b_0)| &= \sum_{0<|k|\leq n} |(f|b_k)| \overset{(1)}{=} \sum_{0<|k|\leq n} \underbrace{\frac{1}{|k|}}_{:= u_k}\underbrace{(f'|b_k)}_{:= v_k} \\
&= \sum_{0<|k|\leq n} u_k v_k \overset{\text{CS-Ugl.}}{\leq} \left( \sum_{0<|k|\leq n} u_k^2 \right)^\frac{1}{2} \left( \sum_{0<|k|\leq n} v_k^2 \right)^\frac{1}{2}\\
&= \left( 2\sum_{k=1}^n u_k^2 \right)^\frac{1}{2} \underbrace{ \left( \sum_{0<|k|\leq n} v_k^2 \right)^\frac{1}{2} }_{ \overset{18.2(3)}{\leq} \|f'\|_2} \\
&\leq \left( 2\sum_{k=1}^\infty u_k^2 \right)^\frac{1}{2} \| f' \|_2
\end{align*}
\end{enumerate}
\end{beweis}

\begin{beispiel}
\begin{enumerate}
\item $f$ sei wie im Beispiel (2) vor \ref{Satz 18.7}. Es war:
\[ f \overset{\| \cdot \|_2}{=} \frac{\pi}{2} - \frac{4}{\pi} \sum_{j=0}^\infty \frac{\cos((2j+1) \cdot )}{(2j+1)^2} \quad \quad \left(\alpha_{2j+1} = \frac{1}{(2j+1)^2}, \alpha_{2j} = 0  \right) \]
Aus \ref{Satz 18.7} bzw. \ref{Satz 18.8} folgt:
\[  f(t) = \frac{\pi}{2} - \frac{4}{\pi} \sum_{j=0}^\infty \frac{\cos((2j+1) t )}{(2j+1)^2} \quad \forall t \in [0,2\pi] \]
Setzt man nun $t=0$, folgt
\[ 0 = \frac{\pi}{2} - \frac{4}{\pi} \sum_{j=0}^\infty \frac{1}{(2j+1)^2} \]
und man erh\"alt durch Umstellen eine Auswertung f\"ur diese eigentlich kompliziert wirkende Reihe:
\[ \sum_{j=0}^\infty \frac{1}{(2j+1)^2} = \frac{1}{1^2} + \frac{1}{3^2} + \frac{1}{5^2} + \ldots = \frac{\pi^2}{8} \]
(dass diese Reihe konvergiert, ist eine einfache \"Ubung aus Ana I; ihren Wert aber haben wir bislang noch nicht berechnet)

\item $f(t) = (t - \pi)^2 \quad (t \in [0,2\pi])$. $f$ ist gerade bzgl. $\pi$, also ist $\beta_k = 0$. Es ist
\[ \alpha_k = \begin{cases} \frac{2}{3}\pi^2, &k=0\\ \frac{4}{k^2}, &k \geq 1 \end{cases} \quad \text{(nachrechnen!)}\]
Also:
\[ f \overset{\| \cdot \|_2}{=} \frac{\pi^2}{3} + 4 \sum_{j=1}^\infty \frac{\cos(j \cdot)}{j^2} \]
Aus \ref{Satz 18.9} bzw. \ref{Satz 18.7}(2) folgt:
\[ f(t) = \frac{\pi^2}{3} + 4 \sum_{j=1}^\infty \frac{\cos(j t)}{j^2}  \quad \forall t \in [0, 2\pi] \]
Setzt man nun $t=0$, erh\"alt man
\[ \pi^2 = \frac{\pi^2}{3} + 4 \sum_{j=1}^\infty \frac{1}{j^2}, \text{ also } \sum_{j=1}^\infty \frac{1}{j^2} = \frac{\pi^2}{6} \]
Damit erh\"alt man z.B. auch
\[ \sum_{j=1}^\infty \frac{1}{(2j)^2} = \frac{1}{4} \sum_{j=1}^\infty \frac{1}{j^2} = \frac{\pi^2}{24} \]
und damit
\[ \sum_{j=1}^\infty \frac{(-1)^{j+1}}{j^2} = \frac{1}{1^2} - \frac{1}{2^2} + \frac{1}{3^2} - \frac{1}{4^2} \pm \ldots = \frac{\pi^2}{8} - \frac{\pi^2}{24} = \frac{\pi^2}{12} \]

\end{enumerate}

\end{beispiel}

\appendix
\chapter{Satz um Satz (hüpft der Has)}
\theoremlisttype{optname}
\listtheorems{satz,wichtigedefinition}

\renewcommand{\indexname}{Stichwortverzeichnis}
\addcontentsline{toc}{chapter}{Stichwortverzeichnis}
\printindex

\chapter{Credits für Analysis III} Abgetippt haben die folgenden Paragraphen:\\% no data in Ana2Vorwort.tex
\textbf{§ 1: $\sigma$-Algebren und Maße}: Rebecca Schwerdt, Peter Pan, Philipp Ost\\
\textbf{§ 2: Das Lebesguemaß}: Rebecca Schwerdt, Philipp Ost\\
\textbf{§ 3: Messbare Funktionen}: Rebecca Schwerdt, Philipp Ost\\
\textbf{§ 4: Konstruktion des Lebesgueintegrals}: Rebecca Schwerdt, Philipp Ost, Peter Pan\\
\textbf{§ 5: Nullmengen}: Rebecca Schwerdt, Jan Ihrens, Philipp Ost\\
\textbf{§ 6: Der Konvergenzsatz von Lebesgue}: Philipp Ost, Jan Ihrens \\
\textbf{§ 7: Parameterintegrale}: Jan Ihrens \\
\textbf{§ 8: Vorbereitungen}: Jan Ihrens \\
\textbf{§ 9: Das Prinzip von Cavalieri}: Jan Ihrens, Rebecca Schwerdt\\
\textbf{§ 10: Der Satz von Fubini}: Jan Ihrens\\
\textbf{§ 11: Der Transformationssatz}: Jan Ihrens, Rebecca Schwerdt\\
\textbf{§ 12: Vorbereitungen für die Integralsätze}: Rebecca Schwerdt\\
\textbf{§ 13: Der Integralsatz von Gau\ss \ im \(\mdr^{2}\)}: Benjamin Unger\\ 
\textbf{§ 14: Flächen im \(\mdr^{3}\)}: Benjamin Unger\\
\textbf{§ 15: Der Integralsatz von Stokes}: Philipp Ost\\
\textbf{§ 16: \(\fl^{p}\)-Räume und \(\mathrm{L}^{p}\)-Räume}: Philipp Ost, Rebecca Schwerdt, Peter Pan, Jan Ihrens \\
\textbf{§ 17: Das Integral im Komplexen}: Peter Pan, Jan Ihrens \\
\textbf{§ 18: Fourierreihen}: Jan Ihrens, Philipp Ost, Peter Pan \\


\fi

\end{document}
